% !TEX encoding = UTF-8 Unicode
% !TEX TS-program = XeLaTeX

%% pagestyle alterations per user request 14 xii 2020

\documentclass{article}
\usepackage{fancyhdr}
\usepackage{multicol}
\usepackage[%
    %a5paper,
    papersize={5.5in,8.5in},
    margin=0.75in,
    top=0.75in,
    bottom=0.75in,
    %twoside
    ]{geometry}
\usepackage{xcolor}
\usepackage{graphicx}

\raggedcolumns
\setlength{\multicolsep}{0pt}
\setlength{\columnseprule}{1pt}

\makeatletter

%% Used for the headnote and in \showit
%% If the text is small it is placed on one line;
%% otherwise it is put into a raggedright paragraph.
\long\def\testoneline#1{%
  \sbox\@tempboxa{#1}%
  \ifdim \wd\@tempboxa <0.75\linewidth
        \begingroup
            \itshape
            #1\par
        \endgroup
  \else
    \parbox{0.75\linewidth}{\raggedright\itshape#1}%
    \par
  \fi
}

\newif\if@mainmatter \@mainmattertrue

%% Borrowed from book.cls
\newcommand\frontmatter{%
    \cleardoublepage
  \@mainmatterfalse
  \pagenumbering{roman}}
\newcommand\mainmatter{%
    \cleardoublepage
  \@mainmattertrue
  \pagenumbering{arabic}}
\makeatother

%% Vary the colors at will

\definecolor{vegcolor}{rgb}{0,0.5,0.2}
\definecolor{frzcolor}{rgb}{0,0,1}
\definecolor{dessertcolor}{rgb}{0.5,0.2,0.1}
\definecolor{makeaheadcolor}{rgb}{0.5,0.5,0.6}
\definecolor{breadcolor}{rgb}{0.9, 0.5, 0.2}

%% Thanks to alephzero for the excellent start:
\newcommand{\recipe}[2][]{%
    \newpage
    \thispagestyle{fancy}
    \lhead{}%
    \chead{}%
    \rhead{}%
    \lfoot{}%
    \rfoot{}%
    \section{#2}%
    \if###1##%
    \else
        \begin{center}
            \testoneline{#1}%
        \end{center}
    \fi
}
\newcommand{\serves}[2][Serves]{%
    \chead{#1 #2}}
\newcommand{\dishtype}[1]{%
    \rhead{#1}%
}
\newcommand{\dishother}[1]{%
    \lhead{#1}%
}
\newcommand{\vegetarian}{%
    {\large\color{vegcolor}\textbf{V}}%
}
\newcommand{\bread}{%
    {\large\color{breadcolor}\textbf{B}}%
}
\newcommand{\freeze}{%
    {\large\color{frzcolor}\textbf{F}}%
}
\newcommand{\dessert}{%
    {\large\color{dessertcolor}\textbf{D}}%
}
\newcommand{\makeahead}{%
    {\large\color{makeaheadcolor}\textbf{M}}%
}
%% Optional arguments for alternate names for these:
\newcommand{\preptime}[2][Prep time]{%
    \lfoot{#1: #2}%
}
\newcommand{\cooktime}[2][Cook time]{%
    \rfoot{#1: #2}%
}
\newcommand{\temp}[1]{%
    $#1^\circ$F}
%% Optional argument is the width of the graphic, default = 1in
\newcommand{\showit}[3][1in]{%
    \begin{center}
        \bigskip
            \includegraphics[width=#1]{#2}%
            \par
            \medskip
            \testoneline{#3}%
            \par
    \end{center}%
}

%% Optional argument for a  heading within the ingredients section
\newcommand{\ingredients}[1][]{%
    \if###1##%
        {\color{red}\Large\textbf{Ingredients}}%
    \else
        \emph{#1}%
    \fi
}

%% Use \obeylines to minimize markup
\newenvironment{ingreds}{%
    \parindent0pt
    \noindent
    \ingredients
    \par
    \smallskip
    \begin{multicols}{2}
    \leftskip1em
    \rightskip0pt plus 3em
    \parskip=0.25em
    \obeylines
    \everypar={\hangindent2em}
}{%
    \end{multicols}%
    \medskip
}

\newcounter{stepnum}

%% Optional argument for an italicized pre-step
%% Also use obeylines to minimize markup here as well
\newenvironment{method}[1][]{%
    \setcounter{stepnum}{0}
    \noindent
    {\color{red}\Large\textbf{Instructions}}%
    \par
    \smallskip
    \if###1##%
    \else
        \noindent
        \emph{#1}
        \par
    \fi
    \begingroup
    \parindent0pt
    \parskip0.25em
        \leftskip2em
    \everypar={\llap{\stepcounter{stepnum}\hbox to2em{\thestepnum.\hfill}}}
}{%
    \par
    \endgroup
    }

\pagestyle{plain}

\begin{document}

\frontmatter
\tableofcontents

\mainmatter

\recipe[Perfect for making Cubano sandwiches]{Cubano Bread}
\serves{4}
\preptime{2 hour}
\cooktime[Cook time]{28--35 minutes}
\dishtype{\vegetarian, \bread}
\dishother{\makeahead}
\begin{ingreds}
    1$\frac{1}{4}$ cups water (\temp{100})
    1 tbsp instant dry yeast
    3$\frac{1}{2}$ cups (500g) unbleached bread flour
    1 tbsp (13g) granulated sugar
    2 tsp (12g) fine sea salt
    2$\frac{1}{2}$ tbsp lard, solidified
% \columnbreak
% \ingredients[For the Crumble Mixture:]
%      80g wholemeal flour
%      80g plain flour
%      80g butter (diced)
%      70g demerara sugar
\end{ingreds}

\begin{method}[Preheat the oven to \temp{400}.]
    In a small bowl, stir together the warm water and yeast. Cover with plastic wrap and let sit for 10 min. The mixture should get slightly foamy, and the yeast should dissolve.

    In a stand mixer bowl, add the flour, sugar, and salt. Mix together thoroughly. With the dough hook attachment, start mixing on medium-low speed. Slowly add the warm yeast mixture and the lard. Mix until combined. Once a cohesive dough is formed, keep mixing for another 3 to 5 minutes, or until smooth.

    Shape the dough into a ball, and place in a greased bowl covered with greased plastic wrap. Let rise at room temp for 45 min to 1 hr, or until doubled in size.

    Punch down the dough to release the gas and place on a lightly floured surface. Divide the dough into 2 even pieces. Cover with a damp towel, and let them rest right where they are for 10 minutes.

    Flatten out 1 piece of dough into about a $\frac{1}{2}$-inch thick rectangle, with the long edge about 10 inches long. From the long edge, tightly roll the dough and close the seams at the bottom and sides.
    Carefully roll the log while applying pressure outward to slightly taper the ends. The log should be 15 inches long. Repeat with the other piece of dough.

    Place the loaves onto a parchment-lined baking sheet 4--6 inches apart. Cover with another baking sheet. Let the dough proof for about 30 minutes at room temp.

    About 10 minutes before the dough is done proofing, bring a 10-inch oven-proof skillet of water to a boil. Preheat the oven to 400°F. Remove the top baking sheet. Using a food-safe spray bottle, lightly spray the dough with water. Using a razor blade or an extremely sharp knife, score a shallow seam along the length of the loaves.

    Place the skillet of boiling water on the bottom rack of the oven. Place the baking sheet with the dough in the middle of the oven on a separate rack above the water. Spray the inside of the oven with a little water to generate steam.

    Let the bread steam for 8--10 minutes. Reduce the heat to \temp{375}, remove the skillet of water, and let the bread bake for another 20--22 minutes, or until lightly browned.

    Let cool completely on a wire rack. Slice and serve. Store on the counter loosely wrapped in a kitchen towel for up to 2 days and then in a resealable bag for 1 more day. Freeze in an airtight container for up to 2 months.


\end{method}

\showit[1.25in]{example-image-b}{This is a picture}

\end{document}