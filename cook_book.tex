\documentclass{article}
\usepackage{enumitem}
\usepackage{fancyhdr}
\usepackage{multicol}
\usepackage[
    % a5paper,
    papersize={5.5in,8.5in},
    margin=0.75in,
    top=0.75in,
    bottom=0.75in,
    twoside
    ]{geometry}
\usepackage{xcolor}
\usepackage{graphicx}

\setlength{\headheight}{14.0pt}

\raggedcolumns{}
\setlength{\multicolsep}{0pt}
\setlength{\columnseprule}{1pt}

\makeatletter

% Decide if you want numbered sections or not.
\setcounter{secnumdepth}{0}

% Used for the headnote and in \showit
% If the text is small it is placed on one line;
% otherwise it is put into a raggedright paragraph.
\long\def\testoneline#1{%
  \sbox\@tempboxa{#1}%
  \ifdim\wd\@tempboxa<0.75 \linewidth{}
        \begingroup
            \itshape{}
            #1\par
        \endgroup
  \else
    \parbox{0.75\linewidth}{\raggedright\itshape#1}
    \par
  \fi
}

\newif\if@mainmatter{} \@mainmattertrue{}

% Borrowed from book.cls
\newcommand\frontmatter{%
    \cleardoublepage{}
  \@mainmatterfalse{}
  \pagenumbering{roman}}
\newcommand\mainmatter{%
    \cleardoublepage{}
  \@mainmattertrue{}
  \pagenumbering{arabic}}
\makeatother

% Vary the colors at will
\definecolor{vegcolor}{rgb}{0,0.5,0.2}
\definecolor{frzcolor}{rgb}{0,0,1}
\definecolor{dessertcolor}{rgb}{0.5,0.2,0.1}
\definecolor{makeaheadcolor}{rgb}{0.5,0.5,0.6}
\definecolor{breadcolor}{rgb}{0.9, 0.5, 0.2}
\definecolor{appcolor}{rgb}{0.2, 0.5, 1.0}
\definecolor{saucecolor}{rgb}{0.2, 0.5, 0.1}

%% Thanks to alephzero for the excellent start:
\newcommand{\recipe}[3][]{
    \newpage
    \thispagestyle{fancy}
    \lhead{}
    \chead{}
    \rhead{}
    \lfoot{}
    \rfoot{}
    \subsection[#2]{\Large#2 $-$ \large#3}
    \if###1##
    \else
        \begin{center}
            \testoneline{#1}
        \end{center}
    \fi
}
\newcommand{\serves}[1]{\chead{Serves #1}}
\newcommand{\dishtype}[1]{\rhead{#1}}
\newcommand{\dishother}[1]{\lhead{#1}}
\newcommand{\vegetarian}{\large\color{vegcolor}\textbf{V}}
\newcommand{\bread}{\large\color{breadcolor}\textbf{B}}
\newcommand{\apps}{\large\color{appcolor}\textbf{A}}
\newcommand{\sauce}{\large\color{saucecolor}\textbf{S}}
\newcommand{\freeze}{\large\color{frzcolor}\textbf{F}}
\newcommand{\dessert}{\large\color{dessertcolor}\textbf{D}}
\newcommand{\makeahead}{\large\color{makeaheadcolor}\textbf{M}}

% Optional arguments for alternate names for these:
\newcommand{\preptime}[2][Prep time]{\lfoot{#1: #2}}
\newcommand{\cooktime}[2][Cook time]{\rfoot{#1: #2}}
\newcommand{\temp}[1]{$#1^\circ$F}

%% Optional argument is the width of the graphic, default = 1in
\newcommand{\showit}[3][1in]{
    \begin{center}
        \bigskip
            \includegraphics[width=#1]{#2}
            \par
            \medskip
            \testoneline{#3}
            \par
    \end{center}
}

%% Optional argument for a  heading within the ingredients section
\newcommand{\ingredients}[1][]{%
    \if###1##{
        \color{red}\Large\textbf{Ingredients}
    }
    \else
        \emph{\textbf{#1:}}
    \fi
}

\newcommand{\recipeSection}[1]{
    \clearpage
    \begin{center}
        \hspace{0pt}\vfill
        \begin{minipage}{\textwidth}
            \centering
            \section[#1]{\huge#1}
        \end{minipage}
        \vfill\hspace{0pt}
    \end{center}
    \clearpage
}

%% Use \obeylines to minimize markup
\newenvironment{ingreds}{
    \parindent0pt
    \noindent
    \ingredients{}
    \par
    \smallskip
    \begin{multicols}{2}
    \leftskip1em
    \rightskip0pt plus 3em
    \parskip=0.25em
    \obeylines{}
    \everypar={\hangindent2em}
}{
    \end{multicols}
    \medskip
}

\newcounter{stepnum}

%% Optional argument for an italicized pre-step
%% Also use obeylines to minimize markup here as well
\newenvironment{method}[1][]{
    \setcounter{stepnum}{0}
    \noindent
    {\color{red}\Large\textbf{Instructions}}
    \par
    \smallskip
    \if#1
    \else
        \noindent
        \emph{#1}
        \par
    \fi
    \begingroup
    \parindent0pt
    \parskip0.25em
        \leftskip2em
    \everypar={\llap{\ensuremath{\stepcounter{stepnum}\hbox to2em{\thestepnum.\hfill}}}}
}{%
    \par
    \endgroup
}

\pagestyle{plain}

\title{Rowden Family Cookbook}
\author{Gage R. Rowden}

\begin{document}

\maketitle

\frontmatter
\tableofcontents

\mainmatter{}

%%%%%%%%%%%%%%%%%%%%%%%%%%%%%%%%%%%%%%%%%%%%%%%%%%%%%%%%%%%%%%

\recipeSection{Breads}

%%%%%%%%%%%%%%%%%%%%%%%%%%%%%%%%%%%%%%%%%%%%%%%%%%%%%%%%%%%%%%

\recipe[Perfect for making Cubano sandwiches.]{Cubano Bread}{Gage}
\serves{4}
\preptime{2 hour}
\cooktime[Cook time]{28--35 minutes}
\dishtype{\vegetarian, \bread}
\dishother{\makeahead}
\begin{ingreds}
    1$\frac{1}{4}$ cups water (\temp{100})
    1 tbsp instant dry yeast
    3$\frac{1}{2}$ cups (500g) unbleached bread flour
    1 tbsp (13g) granulated sugar
    2 tsp (12g) fine sea salt
    2$\frac{1}{2}$ tbsp lard, solidified
% \columnbreak
% \ingredients[For the Crumble Mixture:]
%      80g wholemeal flour
%      80g plain flour
%      80g butter (diced)
%      70g demerara sugar
\end{ingreds}
\begin{method}[Preheat the oven to \temp{400}.]
    In a small bowl, stir together the warm water and yeast. Cover with plastic wrap and let sit for 10 min. The mixture should get slightly foamy, and the yeast should dissolve.\par
    In a stand mixer bowl, add the flour, sugar, and salt. Mix together thoroughly. With the dough hook attachment, start mixing on medium-low speed. Slowly add the warm yeast mixture and the lard. Mix until combined. Once a cohesive dough is formed, keep mixing for another 3 to 5 minutes, or until smooth.\par
    Shape the dough into a ball, and place in a greased bowl covered with greased plastic wrap. Let rise at room temp for 45 min to 1 hr, or until doubled in size.\par
    Punch down the dough to release the gas and place on a lightly floured surface. Divide the dough into 2 even pieces. Cover with a damp towel, and let them rest right where they are for 10 minutes.\par
    Flatten out 1 piece of dough into about a $\frac{1}{2}$-inch thick rectangle, with the long edge $\sim$10'' long.\par
    From the long edge, tightly roll the dough and close the seams at the bottom and sides.\par
    Carefully roll the log while applying pressure outward to \linebreak slightly taper the ends. The log should be $\sim$15'' long.\par
    Repeat with the other piece of dough.\par
    Place the loaves onto a parchment-lined baking sheet 4--6'' apart. Cover with another baking sheet.\par
    Let the dough proof for $>$30 minutes at room temperature.\par
    About 10 minutes before the dough is done proofing, bring a 10'' oven-proof skillet of water to a boil.\par
    Using a spray bottle, lightly spray the dough with water.\par
    Using a razor blade or an extremely sharp knife, score a shallow seam along the length of the loaves.\par
    Place the skillet of boiling water on the bottom rack of the oven.\par
    Place the baking sheet with the dough in the middle of the oven on a separate rack above the water.\par
    Spray the inside of the oven with a little water to generate steam.\par
    Let the bread steam for 8--10 minutes. Reduce the heat to \temp{375}, remove the skillet of water, and let the bread bake for another 20--22 minutes, or until lightly browned.\par
    Let cool completely on a wire rack. Slice and serve. Store on the counter loosely wrapped in a kitchen towel for up to 2 days and then in a resealable bag for 1 more day. Freeze in an airtight container for up to 2 months.\par
\end{method}
% \showit[1.25in]{example-image-b}{This is a picture}

% Browned Butter Cornbread
\recipe[Delicious and savory.]{Browned Butter Cornbread}{Gage}
\serves{4}
\preptime{1 hour}
\cooktime[Cook time]{25 minutes}
\dishtype{\vegetarian, \bread}
\dishother{\makeahead}
\begin{ingreds}
    $\frac{1}{2}$ cup (112 g) unsalted butter
    $\frac{1}{2}$ bunch of sage
    $\frac{1}{2}$ bunch of thyme
    1$\frac{1}{4}$ cups (188 g) all-purpose flour
    $\frac{1}{3}$ cup (67 g) sugar
    3$\frac{1}{2}$ tbsp (47 g) brown sugar
    1$\frac{1}{4}$ tsp (8 g) kosher salt
    1 tbsp (13 g) baking powder
    1$\frac{1}{4}$ cups (175g) finely ground yellow cornmeal
    2 large eggs
    1$\frac{1}{2}$ cups buttermilk
\end{ingreds}
\begin{method}[Preheat an 8” cast-iron skillet in the oven at \temp{400}]
    In a small pot, melt the butter over medium heat. Once melted, continue cooking to brown the butter until it turns dark brown. Then turn off the heat.\par
    Add the sage and thyme, stir to combine, transfer the butter-herb mixture to a separate container, and set aside to cool.\par
    In a bowl, whisk together the flour, sugar, brown sugar, salt, baking powder, and cornmeal.\par
    In a separate bowl, whisk the eggs, then whisk in the buttermilk.
    Discard the herbs from the slightly cooled butter. In a slow stream, whisk the butter into the buttermilk mixture. Then whisk the wet mixture into the dry mixture until completely smooth and combined.\par
    Remove the hot pan from the oven, and lightly grease it with cooking spray. Add the batter to the pan, and spread it evenly.\par
    Bake for 20--25 min, or until a toothpick inserted into the center comes out clean. 
    Cool for a few minutes in the pan and then remove to a cooling rack to cool completely.\par
\end{method}

% Banana Bread
\recipe[This will make anyone's day batter.]{Banana Bread}{Shug}
\serves{4}
\preptime{5 minutes}
\cooktime{15 minutes}
\dishtype{\vegetarian, \sauce}
\begin{ingreds}
    1 red bell pepper
    1 green bell pepper
    10 jalape\~nos
    1$\frac{1}{2}$ cups white vinegar
    $\frac{1}{2}$ tsp salt
    6 cups sugar
    1 pouch liquid fruit pectin   
\end{ingreds}
\begin{method}
    In food processor, finely chop the peppers.\par
    Place peppers in large pot with vinegar, salt, and sugar.\par
    Boil for 10 minutes, stirring often (be careful…it will boil over QUICK!).\par
    Add pectin pouch and boil 1 more minute.\par
    Fill canning jars.\par
    Add to boiling hot water bath for 10 minutes.\par
    Wait a day or so for jelly to set.
\end{method}


%%%%%%%%%%%%%%%%%%%%%%%%%%%%%%%%%%%%%%%%%%%%%%%%%%%%%%%%%%%%%%

\recipeSection{Appetizers \& Snacks}

%%%%%%%%%%%%%%%%%%%%%%%%%%%%%%%%%%%%%%%%%%%%%%%%%%%%%%%%%%%%%%

\recipe[Made at every family get-together.]{Guacamole}{Boss}
\serves{4}
\preptime{15 minutes}
\cooktime[Cook time]{0 minutes}
\dishtype{\vegetarian, \apps}
\begin{ingreds}
    4 Avocados
    $\frac{1}{3}$ onion
    $\frac{1}{2}$ jalapeno
    $\frac{1}{4}$ cup cilantro
    1 tbsp mayo
    2 tsp vinegar
    $\frac{1}{2}$ lemon
    garlic salt
    pepper
    tomatoes (optional)
    stuffed olives (optional)
    cucumbers (optional)
\end{ingreds}
\begin{method}
    Mix and mash all the ingredients together.
\end{method}

%%%%%%%%%%%%%%%%%%%%%%%%%%%%%%%%%%%%%%%%%%%%%%%%%%%%%%%%%%%%%%

\recipeSection{Sauces, Jams, \& Canned Goods}

%%%%%%%%%%%%%%%%%%%%%%%%%%%%%%%%%%%%%%%%%%%%%%%%%%%%%%%%%%%%%%

% Tex-Mex Enchilada Gravy
\recipe[White people can make it too!]{Tex-Mex Enchilada Gravy}{Gage}
\serves{4}
\preptime{5 minutes}
\cooktime{15 minutes}
\dishtype{\vegetarian, \sauce}
\begin{ingreds}
    $\frac{1}{4}$ cup oil or butter
    $\frac{1}{4}$ cup all purpose flour
    $\frac{1}{2}$ tsp ground black pepper
    $\frac{3}{4}$ tsp salt
    2 tsp garlic powder
    2 tsp ground cumin
    $\frac{1}{2}$ tsp dried oregano
    1 tbsp chili powder
    2 cups beef broth
    1 tbsp tomato paste
\end{ingreds}
\begin{method}
    Heat the oil or butter in a medium size skillet over medium heat.
    Add in the flour and stir to mix. Will be a thick mixture.\par
    Allow this to cook for just 1--2 minutes. The roux should be a very light brown.
    Add in the spices. Stir to form a thick paste for no longer than 30 seconds.
    Stir in the broth and tomato paste until smooth.\par
    Continue to heat while stirring for several minutes until the sauce is slightly thickened.\par
    Keep the heat to just below a simmer.\par
    Pour into a jar to store or use to make enchiladas.
\end{method}

% Japanese Curry
\recipe[Try it with the pork katsu.]{Japanese Curry}{Gage}
\serves{4}
\preptime{20 minutes}
\cooktime{}1 hour
\dishtype{\vegetarian}
\begin{ingreds}
    $\frac{1}{4}$ cup of butter (1st)
	$\frac{1}{2}$ cup butter (2nd)
	1$\frac{1}{2}$ tbsp butter (3rd)
	3 onions (julienned)
	1-quart chicken stock
	4 dried shiitake mushroom 
	1 apple (peeled \& grated)
	1 tbsp tomato paste
	$\frac{1}{2}$ cup all-purpose flour
	1 tbsp garam masala
	$\frac{1}{4}$ cup curry powder
	2 tsp MSG
	1 tbsp soy sauce
	2 tbsp Worcestershire sauce
	1 tbsp honey or granulated sugar        
\end{ingreds}
\begin{method}
    Add the 1st butter measurement to a large saucepot and melt over medium-high.\par
    Add the julienned onions and cook for 30 to 45 minutes, stirring often, and adjusting the temperature between medium and low accordingly; if onions start to stick to the bottom, add a splash of water to deglaze.\par
	Bring 1 cup of the chicken stock to a boil and add the shitake mushrooms; after two minutes, remove your mushrooms or until entirely hydrated. Set aside.\par
    Once the onions are done (medium-dark caramel color) add the grated apple to the onions with the tomato paste, stir it and cook for 3 to 4 minutes or until the apple softened. Remove and place it aside.\par
	Over medium-high heat (using the same pot),  add the 2nd butter measurement.\par
    Once melted, add the flour, constantly whisk for 30 seconds.\par
    Add the garam masala (optional), curry powder, and MSG.\par
    Whisk vigorously, and let toast for about 1 minute.\par
	Add back the onions along with the soy sauce, Worcestershire sauce, honey (or sugar), and one quart of chicken shiitake stock, stirring often until thickened\par
    Let simmer for 2 minutes.\par
	Blend the mix until completely smooth, then add the 3rd butter measurement while blending.\par
    Serve with rice and pork katsu.
\end{method}

% Creamy Mexican Dipping Sauce
\recipe[Perfect for dipping a quesadilla.]{Creamy, Spicy Dipping Sauce}{Gage}
\serves{4}
\preptime{5 minutes}
% \cooktime{}
\dishtype{\vegetarian, \sauce}
\begin{ingreds}
    $\frac{2}{3}$ cup of Mexican cream
    hot sauce (I use Valentina\textsuperscript{\textregistered})
    1 lime (juice)
\end{ingreds}
\begin{method}
    Mix all the ingredients together.\par
    Add hot sauce to your desired spice level.
\end{method}

% Boss' Tartar Sauce
\recipe[Enjoy with any fish or just by the spoonful.]{Boss' Tartar Sauce}{Boss}
\serves{4}
\preptime{5 minutes}
% \cooktime{0 minutes}
\dishtype{\vegetarian, \sauce}
\begin{ingreds}
    1 cup real mayo
    $\frac{1}{2}$ cup onion (finely chopped)
    2 tsp dill relish
    2 tsp sweet relish
    $\frac{1}{2}$ jalapeno 
    $\frac{1}{4}$ tsp thyme
    $\frac{1}{2}$ lemon (juice)
    Salt
    Pepper
\end{ingreds}
\begin{method}
    Mix all ingredients together in a bowl.
    Season to taste.
\end{method}

% Katsu Sauce
\recipe[Sweet, tangy, and savory. Perfect for dipping fried food!]{Katsu Sauce}{Gage}
\serves{4}
\preptime{5 minutes}
% \cooktime{0 minutes}
\dishtype{\vegetarian, \sauce}
\begin{ingreds}
    6 tbsp ketchup
    2 tbsp Worcestershire sauce
    2 tbsp oyster sauce
    2 tbsp miso paste (optional)
    2 tsp honey
\end{ingreds}
\begin{method}
    In a small bowl, stir together all of the ingredients until fully incorporated.\par
    Use or store in an airtight container in the refrigerator for up to 1 month.
\end{method}

% Plum Preserves
\recipe[For when your plum tired of life.]{Plum Preserves}{Shug}
\serves{4}
\preptime{5 minutes}
\cooktime{30 minutes}
\dishtype{\vegetarian, \sauce}
\begin{ingreds}
    8 cups plums (pitted/cut)
    4 cups sugar
\end{ingreds}
\begin{method}
    Put plums and sugar in the instant pot.\par
    Saut\'e for on medium-high for 3 minutes.\par
    Pressure cook on high for 1 minute.\par
    Slow release for 10 minutes, then quick release.\par
    Mash and stir on the saut\'e setting until thick but pourable (stirring every 5 minutes).
\end{method}

% Apricot Preserves
\recipe[Perfect pairing for white bread and peanut butter.]{Apricot Preserves}{Shug}
\serves{4}
\preptime{5 minutes}
\cooktime{30 minutes}
\dishtype{\vegetarian, \sauce}
\begin{ingreds}
    8 cups apricots (pitted/cut)
    6 cups sugar
    ¼ cup lemon juice
\end{ingreds}
\begin{method}
    Put apricots and sugar in the instant pot.\par
    Saut\'e on medium-high for 3 minutes.\par
    Pressure cook on high for 1 minute.\par
    Slow release for 10 minutes, then quick release.\par
    Mash and stir on the saut\'e setting until thick but pourable (stirring every 5 minutes).
\end{method}

% Raspberry Jam
\recipe[Try it with chocolate sauce and ice cream.]{Raspberry Jam}{Shug}
\serves{4}
\preptime{5 minutes}
\cooktime{30 minutes}
\dishtype{\vegetarian, \sauce}
\begin{ingreds}
    5 cups raspberries (4 small containers)
    2 cups sugar (more or less to taste)
    Juice from 2 large lemons
    2 tbsp cornstarch
    2 tbsp water    
\end{ingreds}
\begin{method}
    Add the raspberries, sugar, and lemon juice to the instant pot and mix.\par
    Pressure cook on high for 3 minutes.\par
    Slow release for 10 minutes, then quick release. Saut\'e and stir until you reach your desired consistency.\par
    While releasing the pressure, make a slurry with the cornstarch and water.\par
    Add the slurry in small portions to the mixture while stirring until desired viscosity is reached.
\end{method}

% Jalapeno Jam
\recipe[Delicious with cream cheese and crackers!]{Jalape\~no Jelly}{Shug}
\serves{4}
\preptime{5 minutes}
\cooktime{15 minutes}
\dishtype{\vegetarian, \sauce}
\begin{ingreds}
    1 red bell pepper
    1 green bell pepper
    10 jalape\~nos
    1$\frac{1}{2}$ cups white vinegar
    $\frac{1}{2}$ tsp salt
    6 cups sugar
    1 pouch liquid fruit pectin   
\end{ingreds}
\begin{method}
    In food processor, finely chop the peppers.\par
    Place peppers in large pot with vinegar, salt, and sugar.\par
    Boil for 10 minutes, stirring often (be careful…it will boil over QUICK!).\par
    Add pectin pouch and boil 1 more minute.\par
    Fill canning jars.\par
    Add to boiling hot water bath for 10 minutes.\par
    Wait a day or so for jelly to set.
\end{method}

% Pickled Red Cabbage
\recipe[Simple slaw, great for dressing up any recipe.]{Pickled Red Cabbage}{Gage}
\serves{4}
\preptime{15 minutes}
% \cooktime{}
\dishtype{}
\begin{ingreds}
    $\frac{1}{2}$ red cabbage (thinly sliced)
    1$\frac{1}{2}$ cups rice vinegar
    1 cup water
    2 tbsp sugar
    1 tbsp salt
    1 tbsp furikake (optional)
\end{ingreds}
\begin{method}
    Place the sliced cabbage into a large container.\par
    In a medium saucepot, place the vinegar, water, sugar, furikake, and salt.\par
    Bring it to a boil and pour it over the cabbage.\par
    Seal the jar and refrigerate.
\end{method}

%%%%%%%%%%%%%%%%%%%%%%%%%%%%%%%%%%%%%%%%%%%%%%%%%%%%%%%%%%%%%%

\recipeSection{Breakfast}

%%%%%%%%%%%%%%%%%%%%%%%%%%%%%%%%%%%%%%%%%%%%%%%%%%%%%%%%%%%%%%

% Buttermilk Pancakes
\recipe[Make these every Christmas morning.]{Buttermilk Pancakes}{Shug}
\serves{4}
\preptime{5 minutes}
\cooktime{15 minutes}
\dishtype{\vegetarian}
\begin{ingreds}
    2 eggs, separated
    2 cups buttermilk
    2 cups flour
    $\frac{1}{4}$ tsp salt
    $\frac{1}{2}$ cup sugar
    3 tsp baking powder
    $\frac{1}{4}$ tsp baking soda
    2 tbsp vegetable oil
    1 tsp vanilla 
\end{ingreds}
\begin{method}
    Separate eggs, beat egg whites until stiff, and set aside.\par
    Mix egg yolks with buttermilk.\par
    Mix dry ingredients and oil.\par
    Add egg yolks and buttermilk to the dry mixture. Mix well.\par
    Stir in vanilla.\par
    Fold in the egg whites.\par
    Cook on medium-high griddle.
\end{method}

%%%%%%%%%%%%%%%%%%%%%%%%%%%%%%%%%%%%%%%%%%%%%%%%%%%%%%%%%%%%%%

\recipeSection{Side Dishes}

%%%%%%%%%%%%%%%%%%%%%%%%%%%%%%%%%%%%%%%%%%%%%%%%%%%%%%%%%%%%%%

% Thanksgiving Dressing
\recipe[Throwback to the OG Shug.]{Thanksgiving Dressing}{Great Shug}
\serves{4}
\preptime{15 minutes}
\cooktime{45 minutes}
\dishtype{\vegetarian}
\begin{ingreds}
    1 cup chopped onion
    1 cup chopped celery
    1 stick butter (melted)
    2--3 packages baked cornbread
    1 can baked biscuits
    1 container chicken broth
    1 can cream of celery, mushroom, or chicken soup 
\end{ingreds}
\begin{method}[Preheat oven to \temp{375}.]
    Mix cornbread, biscuits, onion, celery, and butter together.\par
    Add chicken broth and soup to get the consistency you want.\par
    Bake until golden brown on top (use the broiler if desired).
\end{method}

% Carrots & Kale
\recipe[For the rabbits in the family.]{Carrots \& Kale}{Shug}
\serves{4}
\preptime{15 minutes}
\cooktime{15 minutes}
\dishtype{\vegetarian}
\begin{ingreds}
    3 large carrots (sliced)
    1 bushel of kale (chopped)
    1 onion (sliced)
    2 tsp garlic salt
    1 tsp lemon pepper
    1 tbsp olive oil
    1 tbsp butter    
\end{ingreds}
\begin{method}
    Sear carrots in olive oil and butter for about 5 minutes.\par
    Add the onion and continue searing with the lid on the pan, without stirring, until carrots are dark and carmelized.\par
    Stir, then add chopped kale, turn off the burner, and cover until kale is wilted.\par
    Add the seasonings to taste.
\end{method}

% Brussels Sprouts
\recipe[These will give lasting luck if consumed on New Year's Eve.]{Brussels Sprouts}{Boss}
\serves{4}
\preptime{15 minutes}
\cooktime{30 minutes}
\dishtype{\vegetarian}
\begin{ingreds}
    2 lbs Brussels sprouts (halved or quartered)
    1 onion (sliced) 
    1 tbsp olive oil
    1 tbsp butter
    2 tsp garlic salt
    1 tsp lemon pepper        
\end{ingreds}
\begin{method}
    Place brussel sprouts flat side down in pan of olive oil and butter.\par
    Add the onion.\par
    Season with garlic salt and lemon pepper.\par
    Cover and sear until brussel sprouts are dark and carmelized.
\end{method}

% Coleslaw
\recipe[For when you want to get slawpy.]{Coleslaw}{Boss}
\serves{4}
\preptime{15 minutes}
% \cooktime{30 minutes}
\dishtype{\vegetarian}
\begin{ingreds}
    1 small head red cabbage
    4 carrots (thinly sliced)
    1 jalapeño (finely diced)
    4 green onions (cut in ribbons)
    1 bunch cilantro (roughly chopped)
    1 cup REAL mayo
    1 tsp cumin
    1 tsp smoked paprika
    2 tsp garlic salt
    black pepper (to taste)
    1 tsp vinegar       
\end{ingreds}
\begin{method}
    Mix all the ingredients together in a bowl.
\end{method}

% White Rice
\recipe[Simple and goes with literally everything.]{White Rice}{Gage}
\serves{4}
\preptime{5 minutes}
\cooktime{25 minutes}
\dishtype{\vegetarian}
\begin{ingreds}
    1 cup white rice
    2 cups water
    1 tsp Kosher salt          
\end{ingreds}
\begin{method}
    In a fine mesh strainer add the rice.\par
    Place the strainer into a bowl, and fill with water.\par
    Mix the rice in the water and dispose of the water.\par
    Repeat this process until the water no longer becomes cloudy.\par
    Add the rice to a medium sauce-pot along with 2 cups of room temperature water and salt.\par
    On the smallest burner, bring to a boil, then reduce the heat to low and cover.\par Simmer for 20 minutes.
\end{method}



%%%%%%%%%%%%%%%%%%%%%%%%%%%%%%%%%%%%%%%%%%%%%%%%%%%%%%%%%%%%%%

\recipeSection{Mains}

%%%%%%%%%%%%%%%%%%%%%%%%%%%%%%%%%%%%%%%%%%%%%%%%%%%%%%%%%%%%%%

% Fettuccine Alfredo
\recipe[It slaps.]{Fettuccine Alfredo}{Jacy}
\serves{4}
\preptime{5 minutes}
\cooktime{25 minutes}
\dishtype{}
\begin{ingreds}
    $\frac{1}{2}$ cup butter
    1 $\frac{1}{2}$ cup heavy whipping cream
    2 tsp minced garlic
    $\frac{1}{2}$ tsp italian seasoning
    $\frac{1}{2}$ tsp salt
    $\frac{1}{4}$ tsp pepper
    2 cups grated parmesan cheese
    1 bag fettuccine noodles
    1 lb chicken breast or shrimp
\end{ingreds}
\begin{method}
    Add butter and cream to a large skillet.\par
    Simmer over low for 2 minutes.\par
    Whisk in garlic and seasonings for one minute.\par
    Whisk in parmesan until melted.\par
    Mix in with noodles and your choice of chicken or shrimp.\par
    Simmer until the meat is cooked.
\end{method}

% Chili
\recipe[Make this every fall.]{Chili}{Boss}
\serves{4}
\preptime{5 minutes}
\cooktime{2--4 hours}
\dishtype{}
\begin{ingreds}
    1 lb ground beef
    1 onion
    3 garlic cloves
    2 jalapenos 
    1 can stewed tomatoes (16 oz.)
    1 can tomato sauce (16 oz.)
    3 tbsp chili powder (at least)
    1 tbsp pepper
    1 tbsp cumin
    1 tbsp garlic salt
    1 tsp paprika
    $\frac{1}{4}$ tsp coriander
    2 tsp rosemary
    2 cups water
\end{ingreds}
\begin{method}
    Saute onion, garlic, and jalapenos with olive oil.\par
    In a separate pan, brown the meat.\par
    Once browned, drain the grease, and add to the sauteed vegetables.\par
    Add all other ingredients.\par
    Simmer for 2--4 hours.\par
    Seasoning does not have to be exact, just add to your liking. 
\end{method}

% Chili 2.0
\recipe[Gage's rendition on Boss' chili.]{Chili 2.0}{Gage}
\serves{4}
\preptime{5 minutes}
\cooktime{2--4 hours}
\dishtype{}
\begin{ingreds}
    1 tbsp peanut oil
    1 lb ground beef
    1 lb breakfast sausage
    4 strips of bacon sliced 
    1 onion finely diced
    3 garlic cloves crushed
    3 jalape\~nos finely diced
    $\frac{1}{4}$ cup flour
    $\frac{1}{3}$ cup red wine (optional)
    1 large can crushed tomatoes
    $\geq$3 tbsp chili powder
    1 tbsp pepper
    1 tbsp cumin
    1 tbsp garlic powder
    1 tsp paprika
    $\frac{1}{4}$ tsp coriander
    2 tsp rosemary
    Salt (to taste)
    2 cups water
\end{ingreds}
\begin{method}
    In a large pot, add the oil and heat over high until shimmering.\par
    Add the bacon and cook until lightly browned.\par
    Add the beef and breakfast sausage.\par
    Cook over high until browned.\par
    Remove the meat and add the onions and jalape\~nos.\par
    Cook until translucent.\par
    Add the flour and thoroughly mix.\par
    Cook until lightly browned.\par
    Add the garlic, and continue to mix for another 30 sec.\par
    Deglaze with the wine or beef stock, then add the crushed tomatoes and all the spices.\par
    Mix thoroughly, and add water to desired thickness.\par
    Cover, reduce heat to low, and simmer for 2--4 hours. 
\end{method}

% Blackened Catfish
\recipe[Simple, easy, delicious.]{Blackened Catfish}{Boss}
\serves{4}
\preptime{5 minutes}
\cooktime{10 minutes}
\dishtype{}
\begin{ingreds}
    2 catfish filets
    1$\frac{1}{2}$ tsp garlic salt
    1$\frac{1}{2}$ tsp paprika  
    black pepper (to taste)  
\end{ingreds}
\begin{method}
    Season fish to your liking on both sides. Use pepper generously.\par
    Cook on medium-high heat in olive oil for approximately 3 minutes on each side, or until done. Cast iron skillet will produce the best results. 
\end{method}

% Tomato Soup
\recipe[Of course you have to eat it with a grilled cheese!]{Tomato Soup}{Shug}
\serves{4}
\preptime{10 minutes}
\cooktime{30 minutes}
\dishtype{}
\begin{ingreds}
    $\frac{1}{2}$ onion
    6 tbsp butter
    1 can diced tomatoes
    1 can tomato sauce
    $\frac{1}{4}$ cup chicken broth
    3 tbsp sugar
    1 cup half \& half
    salt and pepper (to taste)
    1$\frac{1}{2}$ tsp parsley
    1$\frac{1}{2}$ tsp basil    
\end{ingreds}
\begin{method}
    Saut\'e onions in the butter.\par
    Process the sauteed onions and diced tomatoes in a food processor.\par
    Put back in the pot to mix with tomato sauce and chicken broth.\par
    Bring soup to almost a simmer.\par
    Add sugar, spices, and half \& half.
\end{method}

% Tomato Bisque
\recipe[If you liked the last one, you'll love this!]{Tomato Bisque}{Gage}
\serves{4}
\preptime{20 minutes}
\cooktime{40 minutes}
\dishtype{}
\begin{ingreds}
    6 tbsp butter
    1 tbsp olive oil
    1 medium onion, diced
    1 carrot, diced
    1 stick of celery, diced
    1 red bell pepper, diced
    3 cloves of garlic, diced
    1 can of San Marzano crushed tomatoes (28 oz)
    chicken stock (to desired thickness)
    1 cup heavy cream
    $\frac{1}{2}$ cup sherry (optional)
    1 tsp black pepper
    salt to taste
    sugar to taste
    basil (chiffonade)
    $\frac{1}{2}$ cup balsamic vinegar   
\end{ingreds}
\begin{method}
    Add the butter and oil to a saut\'e pan.\par
    Saut\'e (with a pinch of salt) the onion, carrot, celery, and bell pepper on medium-high heat until the onions are translucent.\par
    Add the garlic and cook for an additional 30 seconds.\par
    Deglaze the pan with the sherry (or chicken stock).\par
    Add the tomatoes and chicken stock and bring to a simmer.\par
    Add the black pepper and sugar until the desired taste is reached.\par
    Simmer for $>$30 min.\par
    Transfer to a blender and blend on high until completely \linebreak smooth.\par
    Transfer back to the saut\'e pan and add in the heavy cream.\par
    Add the balsamic vinegar to a non-stick pan on medium heat to make a reduction.\par
    Serve with the chiffonade basil and the balsamic reduction.
\end{method}

% Chicken Enchilada Soup
\recipe[This just might cure your cold.]{Chicken Enchilada Soup}{Shug}
\serves{4}
\preptime{15 minutes}
\cooktime{1 minutes}
\dishtype{}
\begin{ingreds}
    2 tbsp olive oil
    1 small onion (diced)
    2 stalks celery (diced)
    1 garlic clove (sliced)
    1 can green chilis (4 oz)
    1 can Rotel tomatoes
    1 can beef broth (16 oz)
    1 can chicken broth (16 oz)
    1 can cream of chicken soup
    1$\frac{1}{2}$ cups water
    1 tbsp A1 steak sauce
    $\frac{1}{4}$ tsp pepper
    2 tbsp Worcestershire sauce
    1 tbsp cumin
    1 tbsp chili powder
    2--3 cups cooked white rice 
    1 rotisserie chicken (pulled) 
\end{ingreds}
\begin{method}[For cooked rice, see the recipe in the side dish section.]
    In a large pot, saut\'e the onion and garlic in olive oil.\par
    Add all of the ingredients to the pot, excluding the rice.\par
    Bring to a boil, then lower the heat and simmer for 1 hour.\par
    Add the cooked rice, and serve.
\end{method}

% Chicken Orzo Soup
\recipe[Why fuss with rice, if you can just use orzo?]{Chicken Orzo Soup}{Gage}
\serves{4}
\preptime{20 minutes}
\cooktime{45 minutes}
\dishtype{}
\begin{ingreds}
    1 tbsp olive oil
    6 garlic cloves (minced)
    1 yellow onion (diced)
    2 carrots (thinly sliced)
    2 celery stalks (chopped)
    1 tbsp fresh ginger
    1 tsp cayenne
    1 tsp ground turmeric
    6 cups chicken broth
    1 lb chicken breast/thighs
    1 tsp chopped rosemary
    1 tsp chopped thyme
    $\frac{1}{2}$ tsp salt
    black pepper (to taste)
    1 cup orzo pasta
    $\frac{2}{3}$ cup frozen peas
\end{ingreds}
\begin{method}[Feel free to substitute with pre-cooked chicken.]
    Place a large dutch oven over medium high heat.\par
    Add the olive oil once the pot is ripping hot and add in the chicken making sure not to crowd the pot.\par
    Get a good sear on all sides of the chicken and save for later. There should be a nice fond in the pan at this point.\par
    Add in the onion, carrots and celery.\par
    Cook for a few minutes until the onion becomes translucent.\par
    Add in the ginger, garlic, and turmeric and saut\'e for 30 seconds.\par
    Add in chicken broth, chicken breast from earlier, and seasonings.\par
    Bring soup to a boil then reduce heat to low and simmer until chicken is fully cooked.\par
    Once chicken is cooked, remove and transfer to a cutting board to shred.\par
    Add chicken back to the pot and stir in frozen peas and orzo.\par
    Cook for another 7--8 minutes until orzo is just barely tender.
\end{method}

% Chicken Tacos
\recipe[No need to stress yourself out with these other fancy recipes.]{Chicken Tacos}{Shug}
\serves{4}
\preptime{5 minutes}
\cooktime{4--6 hours}
\dishtype{}
\begin{ingreds}
    4 chicken breasts
    1 chicken taco seasoning packet
    1 jar Kylito's salsa
    1 can corn, drained (optional)
    2 small cans green chiles (optional)
\end{ingreds}
\begin{method}[Feel free to substitute with pre-cooked chicken.]
    Place chicken in a crockpot.\par
    Sprinkle chicken taco seasoning packet over all the chicken.\par
    Pour in salsa, corn, and green chiles.\par
    Cook on high for 4 hours, or low for 6 hours.\par
    Shred and stir. Serve with tortillas.
\end{method}

% Carne Asada
\recipe[Throw these on a quesadilla, and you'll be in heaven.]{Carne Asada}{Gage}
\serves{4}
\preptime{20 minutes}
\cooktime{10--15 minutes}
\dishtype{}
\begin{ingreds}
    2 lbs skirt steak
    olive oil (to coat the steak)
    2 tbsp soy sauce
    1 orange (juice and zest)
    2 limes (juice and zest)
    6 cloves garlic (crushed)
    1 handful of cilantro (chopped)
    $\frac{2}{3}$ tbsp chili powder
    1 tbsp garlic powder
    2 tsp cumin
    1 heavy pinch oregano
    $\frac{2}{3}$ tbsp kosher salt
\end{ingreds}
\begin{method}[A flame grill is ideal, but a ripping hot flat iron will also work.]
    In a bowl, add the skirt steak, and coat with olive oil.\par
    Add the soy sauce, citrus, garlic, cilantro, and seasonings, and mix together to fully coat the steak.\par
    Marinate in the refrigerator for $>$1 hour.\par
    Add the steaks to the grill, and allow each side to get some char marks ($~$5 minutes on each side).\par
    Remove the steaks from the grill and let rest for $>$5 minutes.\par
    Cut the steaks against the grain, and serve. 
\end{method}

% Chicken Spaghetti
\recipe[Always a popular choice in our house.]{Chicken Spaghetti}{Shug}
\serves{4}
\preptime{10 minutes}
\cooktime{45 minutes}
\dishtype{}
\begin{ingreds}
    1 package of spaghetti
    3 chicken breasts
    2 cups chicken broth
    $\frac{1}{2}$ stick butter 
    1 buttermilk ranch packet
    $\frac{1}{2}$ onion, diced
    1 can rotel tomatoes 
    1 can cream of chicken soup
    garlic salt
    black pepper
    1 block cream cheese
    $\frac{1}{3}$ block Velveeta\textsuperscript{\textregistered} cheese
\end{ingreds}
\begin{method}
    Add all ingredients to the crock pot, excluding the cream cheese and Velveeta\textsuperscript{\textregistered}.\par
    Cook on high for 3$\frac{1}{2}$ hours.\par
    Add cream cheese and Velveeta\textsuperscript{\textregistered}, then cook an additional 30 minutes.\par
    Shred and stir.\par
    Boil the spaghetti to desired tenderness, strain, and add to the mixture in the crock pot. 
\end{method}

% Spaghetti-Stuffed Sandwiches
\recipe[As delicious as it is unhinged.]{Spaghetti-Stuffed Sandwiches}{Jacy}
\serves{4}
\preptime{10 minutes}
\cooktime{30 minutes}
\dishtype{}
\begin{ingreds}
    1 pound ground beef
    1 can tomato sauce
    2$\frac{1}{2}$ cups beef broth
    16 oz spaghetti 
    Italian loaf bread
    $\frac{1}{4}$ cup butter
    $\frac{1}{2}$ tbsp minced garlic (for compound butter)
    1 tbsp minced garlic (for sauce)
    garlic salt
    black pepper
    Italian seasoning
\end{ingreds}
\begin{method}[Preheat the oven to a broil.]
    Cut the ends off of the Italian loaf bread and hollow them out.\par
    Melt the butter, and add $\frac{1}{2}$ tbsp of minced garlic, italian seasoning, and salt and pepper to the melted butter.\par
    Use a brush to spread the butter mixture on the outside of the hollowed out Italian loaf, and set aside.\par
    Brown the ground beef in a pot on high heat.\par
    Add the tomato sauce, beef broth, 1 tbsp minced garlic, and seasonings, and stir.\par
    Add the spaghetti, and cover the pot.\par
    Cook on a simmer until spaghetti is done.\par
    Stir every now and then to ensure it doesn't stick.\par
    Break up the pieces of spaghetti when you stir.\par
    Put your Italian loaves under the broiler, cut-side-up, to toast them.\par
    Stuff the hollowed out loafs with the spaghetti. 
\end{method}

% Spaghetti and Meatballs
\recipe[A more classic use of spaghetti.]{Spaghetti and Meatballs}{Gage}
\serves{4}
\preptime{30 minutes}
\cooktime{30 minutes}
\dishtype{}
\begin{ingreds}
    \ingredients[Meatballs]
        $\frac{1}{4}$ lb chopped mortadella
        1 lb ground beef
        1 tsp (2g) finely ground fennel
        1$\frac{3}{4}$ tsp (12g) fine sea salt
        3 cloves garlic (diced)
        $\frac{1}{4}$ cup (20g) freshly grated parmesan
        black pepper to taste
        $\frac{1}{2}$ cup (35g) panko bread crumbs
        1 whole egg
    \columnbreak{}
    \ingredients[Sauce]
        $\frac{1}{4}$ cup extra virgin olive oil
        4 cloves garlic, thinly sliced
        1 tsp (2g) red pepper flakes
        1 can crushed tomatoes (28 oz)
        1 bunch of basil
        freshly grated Parmigiano for serving
        chiffonade basil for serving
        1 package spaghetti
    
\end{ingreds}
\begin{method}[Prepare the meatballs before making the sauce.]
    In a bowl, place your chopped mortadella, combine it with the ground beef, fennel seed, salt, garlic, parmesan cheese, and black pepper, mix it well, and add the panko and eggs and mix again until emulsified and tacky.\par
    Use a large cookie scoop to make the meatballs, around twenty, place them on a sheet tray and roll them into balls.\par
    In a large saut\'ee pan, over medium-high, pour enough olive oil to cover the bottom of the pan; once the oil is hot, add all the meatballs in one single layer, and sear for about two minutes, flip and sear one or two more sides, until golden brown (it's okay if there are not cooked all the way).\par
    Remove the meatballs from the pan and reserve.\par

    In the same pan where the meatballs were cooked, reduce the heat to medium, add the garlic, and saut\'e for about 5 minutes.\par
    Add the pepper flakes, saut\'e for thirty seconds and add your crushed tomatoes.\par
    Stir in a pinch of sugar (or to desired sweetness).\par
    Add the meatballs back into the pan, bring to a simmer, and reduce the heat to medium-low and simmer for five to eight minutes.\par
    Halfway through this process, add the basil leaves and let simmer until the meatballs cook all the way through.\par

    Place spaghetti in a pot of boiling water that has been seasoned generously with salt.\par
    Cook according to package instructions or until done.\par
    Using tongs, pick up the pasta, let it drain slightly, add it to your sauce until all your pasta has been added.\par
    Place one portion of pasta in a shallow bowl, pour some sauce on top and two or three meatballs, grate some fresh parmesan, and finally, some chiffonade fresh basil.
\end{method}

% Lemon Garlic Shrimp Pasta
\recipe[Lemon, garlic, shrimp? Name a better combo!]{Lemon Garlic Shrimp Pasta}{Boss}
\serves{4}
\preptime{5--15 minutes}
\cooktime{15 minutes}
\dishtype{}
\begin{ingreds}
    1 lb shrimp
    1 lemon (juice)
    1 tbsp minced garlic (or 4 garlic cloves)
    2 tbsp butter (for saut\'eing)
    4 tbsp butter (for sauce)
    2 tbsp water
    pasta of choice
    salt (to taste)
    black pepper (to taste)    
\end{ingreds}
\begin{method}
    Melt 2 tbsp butter in a large skillet on medium-high heat.\par
    Add the garlic and cook for about 1 minute.\par
    Fry the shrimp, cooking for 2 minutes on each side.\par
    Add salt and pepper to taste and stir occasionally.\par
    Add in the remaining butter, lemon juice, and water.\par
    Cook until the butter melts and the shrimp have cooked fully.\par
    Take off heat, add more salt and pepper if needed, and serve on your preferred pasta.
\end{method}

% This recipe needs more detail.
% % Fish Tacos
% \recipe[Eat it on the go! I don't care!]{Fish Tacos}{Boss}
% \serves{4}
% \preptime{5--15 minutes}
% \cooktime{15 minutes}
% \dishtype{}
% \begin{ingreds}
%     Catfish 
%     Garlic salt
%     Pepper
%     Cumin
%     Cayenne pepper
%     Paprika
%     Dried onion
%     Olive oil
%     Butter      
% \end{ingreds}
% \begin{method}
% Season catfish with all seasonings except dried onion, and sear in olive oil - 4 minutes on each side.  Chop up in the pan, mixing in butter and dried onion.  Garnish with Boss’s cole slaw, fresh chopped onion, and fresh cilantro.
% \end{method}

% Chorizo Tacos with Plantains
\recipe[This one is worth the effort. I promise!]{Chorizo Tacos with Plantains}{Gage}
\serves{4}
\preptime{15 minutes}
\cooktime{30 minutes}
\dishtype{}
\begin{ingreds}
    1 link of chorizo
    1 plantain
    1 lime (juice)
    $\frac{1}{2}$ onion (diced)
    $\frac{1}{2}$ onion (Julienned)
    1 bunch cilantro (chopped)
    queso Oaxaca (or mozzarella)
    6--8 corn tortillas
    vegetable oil
    $\frac{1}{4}$ cup apple cider vinegar
    hot sauce (like Cholula\textsuperscript{\textregistered})
    Mexican lager (like Corona\textsuperscript{\textregistered}) (or water)  
\end{ingreds}
\begin{method}
    In a small container, add the apple cider vinegar, juice of half a lime, and the Julienned onions. Seal and shake briefly and set aside to pickle.\par
    Shred the queso Oaxaca or mozzarella. You won't need much.\par
    Slice the plantain at a bias into $\frac{1}{4}$'' slices.\par
    Heat a non-stick skillet over medium-high heat and add the chorizo. Cook until browned.\par
    Remove the chorizo, and add the onions until translucent.\par
    Deglaze with a $\frac{1}{4}$ cup of the lager (or water) until it boils off, then add back the chorizo.\par
    Mix briefly, then set on the back burner.\par
    In a large deep skillet add the oil so that it just covers the bottom and then some.\par
    Once the oil is shimmering, add the plantain slices in batches, making sure not to overcrowd the pan.\par
    Cook until one side is browned then flip. Once done, set aside.\par
    Heat a non-stick skillet over medium heat until just smoking.\par
    Add the tortillas, and cook both sides so that they are just slightly burned in a few spots.\par
    Store in a bowl covered with a rag, or they will get stale.\par
    Now plate. Add the tortillas, then the cheese. You can melt with a blowtorch or in the microwave. Add the chorizo, then 2--3 slices of plantains, then the pickled onions. Cut a few leaves of cilantro on top with scissors, and spritz with a quarter of lime. Optionally, top with a few dabs of hot sauce.
\end{method}

% Cheesy, Bacon Chicken
\recipe[This is just sinful, but we can keep it between us.]{Cheesy Bacon Chicken}{Boss}
\serves{4}
\preptime{15 minutes}
\cooktime{40 minutes}
\dishtype{}
\begin{ingreds}
    Cheddar cheese (slices \& grated)
    2 large chicken breasts
    1 block cream cheese (softened)
    1 can corn
    1 jalape\~no (chopped)
    $\frac{1}{2}$ onion (chopped)
    1 clove garlic (chopped)
    6 strips bacon
\end{ingreds}
\begin{method}[Preheat oven to \temp{375}.]
    Cut several deep slits lengthwise into the chicken breasts.\par
    Insert bacon and cheddar slices inside each slit.\par
    Mix all other ingredients.\par
    Cover chicken with mixture.\par
    Bake for 40 minutes in a cast iron skillet.
\end{method}

% Ground Beef Bulgogi
\recipe[Sweet and savory, and good in a pinch.]{Ground Beef Bulgogi}{Gage}
\serves{4}
\preptime{15 minutes}
\cooktime{20 minutes}
\dishtype{}
\begin{ingreds}
    \ingredients[Bulgogi Sauce]
        $\frac{1}{2}$ cup soy sauce
        2 tbsp rice wine vinegar
        1 tbsp + 2 tsp brown sugar
        2 tsp sesame oil
        1 tsp ginger (minced)
        1 tsp garlic (minced)    
    \columnbreak{}
    \ingredients[Ground Beef]
        1 lb ground beef
        $\frac{1}{2}$ yellow onion (diced)
    \smallbreak{}
    \ingredients[Toppings]
        1 tbsp scallion (diced)
        1 tsp sesame seeds
        4 large soft boiled eggs
\end{ingreds}
\begin{method}[Prepare the meatballs before making the sauce.]
    In a medium bowl combine the ingredients for the bulgogi sauce.\par
    Mix well to combine and set aside.\par
    In a skillet over medium-high heat, add ground beef and onion and cook until browned (about 5 minutes). The meat will still be a little pink.\par
    Add the Bulgogi sauce and mix well.\par
    Simmer for 6--10 more minutes, and remove from heat.\par
    Serve on top of rice.\par
    Top with scallions, toasted sesame seeds, and a soft boiled egg.
\end{method}

% Pork Katsu
\recipe[It's deep-fried pork. What else needs to be said?]{Pork Katsu}{Gage}
\serves{4}
\preptime{20 minutes}
\cooktime{10 minutes}
\dishtype{}
\begin{ingreds}
    4 boneless pork loin chops
    salt to taste
    1 pinch MSG
    $\frac{1}{2}$ cup all-purpose flour
    2 eggs plus a splash of water
    1$\frac{1}{2}$ cups panko breadcrumbs
    3--4 cups oil for frying
\end{ingreds}
\begin{method}
    Cut the meat off the bone, and score the fat cap down, so the meat does not curl while cooking.\par
    With a sharp knife, score the meat lightly, on the exposed flesh, in a crosshatch pattern, no more than $\frac{1}{4}$'' deep.\par
	With a meat mallet, flatten the pork loin portions until they are $\frac{1}{2}$'' thick; season with salt and a pinch of MSG.\@\par
    Place in the fridge for 30 minutes or overnight.\par
	Set up your breading station by placing the flour, beaten eggs and panko, in separate bowls.\par
    Toss the meat in the flour, coating completely, and shake off the excess\par
    Dip in the egg wash and let it drain slightly\par
    Press it into the panko and make sure everything is well covered.\par
    Repeat with all the chops.
	Fill up a large pot with 2$\frac{1}{2}$'' of vegetable oil and heat to \temp{340}.\par
    Fry the pork chops, one or two at a time for 4--6 minutes or until golden brown and crisp ($>$\temp{140} internal).
\end{method}

% Mojo Pulled Pork
\recipe[Make this for every party you throw.]{Mojo Pulled Pork}{Gage}
\serves{4}
\preptime{20 minutes}
\cooktime{3$\frac{1}{2}$--4 hours}
\dishtype{}
\begin{ingreds}
    1 pork shoulder
    Chicken stock (if needed)
    Kosher salt (to taste)
    1 onion (quartered)
    2 heads of garlic (peeled)
    2 oranges (juice and zest)
    3 limes (juice and zest)
    2 tbsp (7g) oregano leaves
    $\frac{1}{2}$ bunch of mint leaves
    1 tbsp (5g) ground cumin
    2 serrano chilies
    1 cup extra virgin olive oil
    1$\frac{1}{2}$ tbsp kosher salt
    1 cup fresh lime juice
    1 cup fresh orange juice or pineapple juice
\end{ingreds}
\begin{method}[Marinate overnight; Preheat the oven to \temp{400} on day of cooking.]
    In a blender, add the onion, garlic, orange zest, lime zest, oregano leaves, mint leaves, cumin, serranos, olive oil, salt, lime juice, and orange juice.\par
    Blend together on high speed until completely smooth.\par
    Reserve 1 cup of the marinade to use for dipping.\par
    With a paring knife, poke deep holes all over the pork shoulder.\par
    In a large resealable bag, place the pork shoulder, and pour in the remaining mojo marinade to cover the meat. Seal the bag and marinate in the refrigerator overnight.\par
    Remove the pork from the marinade and place in a dutch oven.\par
    Pour in all of the marinade. The marinade should come about halfway up the pot, but if not, add a little bit of chicken stock.\par
    Braise the pork, uncovered, for 20 minutes. Then reduce the temperature to 350°F and cook until the internal temperature of the pork is 200°F ($~$3$\frac{1}{2}$--4 hours) flipping occasionally.\par
    Remove the pork from the dutch oven and place on a cutting board to cool.\par
    In a large bowl, using two forks, shred the meat.\par
    Toss together using the braising liquid to coat the meat to your desired level of fattiness.\par
    Season to taste with salt.    
\end{method}

% Chicken Katsu
\recipe[It's fried chicken my boy.]{Chicken Katsu}{Gage}
\serves{4}
\preptime{20 minutes}
\cooktime{15 minutes}
\dishtype{}
\begin{ingreds}
    1$\frac{1}{2}$ cups all-purpose flour
    3 large eggs
    2 tbsp water
    2 cups panko
    4 boneless, skinless chicken breasts/thighs
    2 cups high-heat oil
    kosher salt
\end{ingreds}
\begin{method}
    Place the chicken breasts between two pieces of plastic wrap and pound to an even thickness of $~\frac{1}{2}$''. Season with salt.\par
    Prepare 3 separate shallow bowls for the breading station. In the first, place the flour. In the second, whisk together the eggs and water. In the last, place the panko.\par
    Thoroughly coat each piece of chicken in the flour, shaking off any excess.\par
    Coat evenly in the egg, making sure no dry spots remain.\par
    Finally, coat the chicken on all sides in the panko, pressing to make sure it's thoroughly coated.\par
    Set aside on a piece of parchment paper, and separate each breast with a new layer of parchment.\par
    Fill a large deep cast-iron skillet or heavy-bottomed pot with the oil, and heat over high heat until just shimmering, or about \temp{325--350}.\par
    Carefully place the chicken breast in the oil one at a time.\par
    Fry for 3--5 minutes, or until crispy golden brown.\par
    Flip and fry for an additional 3--5 minutes, or until it reaches an internal temp of 165°F.\par
    Place on a wire rack in a rimmed baking sheet, season again with salt and let drain and cool.\par Repeat with the remaining chicken breasts.\par
    Serve sliced, with a side of spicy cabbage slaw, and drizzled with katsu sauce and Sriracha.
\end{method}

% Butter Chicken
\recipe[Raleigh's favorite meal of all time.]{Butter Chicken}{Gage}
\serves{4}
\preptime{30 minutes}
\cooktime{30 minutes}
\dishtype{}
\begin{ingreds}
    \ingredients[Marinade]
        1$\frac{1}{2}$ lbs chicken thighs (1'' pieces)
        $\frac{3}{4}$ cup (145g) yogurt
        2$\frac{1}{2}$ tsp (6g) chili powder
        1 tsp (2g) turmeric
        2 tsp lemon juice
        1 tsp (2g) cumin
        1 tbsp (7g) Garam masala
        2 teaspoons (10g) fine sea salt
        2-inch knob ginger
    \columnbreak{}
    \ingredients[Curry Sauce]
        1 onion, thinly sliced
        5 garlic cloves, chopped
        1 tsp (2g) garam masala
        1 tsp (2g) chili powder
        $\frac{1}{2}$ tsp (1g) turmeric
        $\frac{1}{2}$ tsp (1g) fenugreek
        1 tbsp (13g) sugar
        2 tbsp (28g) ghee
        1 tbsp (15g) grated ginger
        1 large can crushed tomatoes
        $\frac{3}{4}$ cup (177ml) heavy cream
        2 tbsp (28g) unsalted butter
        fresh cilantro for garnish
        salt to taste
\end{ingreds}
\begin{method}[Requires $>$30 minutes of marination.]
    Place your chicken thighs in a bowl along with the rest of the marinade ingredients and mix thoroughly, ensuring the chicken gets completely coated with the marinade.\par
    Cover with plastic wrap and reserve in the refrigerator.\par
    Heat a 12'' nonstick skillet over medium-high heat and add the ghee.\par
    Once melted, add your marinated chicken pieces and sear them on all sides until golden brown. It is okay if it does not get cooked all the way through.\par
    Remove your chicken, place it in a large bowl, and reserve.\par
    Add the onions, season with salt, and cook until just translucent.\par
    Add the garlic and ginger and cook for another 30 seconds.\par
    Stirring often, add the garam masala, chili powder, turmeric, fenugreek, and sugar.\par
    Add in your crushed tomatoes and reduce for 2 minutes.\par
    Place the onion mix in the blender and blend on high until smooth.\par
    Place the sauce back in the pan, add your chicken, cover with a lid and simmer for seven minutes.\par
    Add the heavy cream and simmer until thickened.\par
    Cut the heat, stir in butter, and season with salt to taste.\par
    Serve your chicken over rice, and garnished with cilantro.    
\end{method}

% Oven Swiss Steak
\recipe[A classic enjoyed during harvest.]{Oven Swiss Steak}{Gran}
\serves{4}
\preptime{20 minutes}
\cooktime{1 hour}
\dishtype{}
\begin{ingreds}
    4 round steaks
    2 carrots (chopped)
    2 celery stalks (chopped)
    $\frac{1}{2}$ onion (chopped)
    1 can stewed tomatoes
    1--2 cloves garlic (minced)
    1 can cream of mushroom soup
    beef broth (or water)
    flour with favorite seasonings mixed in
\end{ingreds}
\begin{method}[Preheat oven to \temp{300}.]
    Coat the steaks evenly with the seasoned flour.\par
    Add the steaks to a ripping hot pan and brown on each side in olive oil (about 5 minute each side).`\ '\par
    Place in a Pyrex\textsuperscript{\textregistered} dish.\par
    Mix the fresh vegetables together and pour over the steaks.\par
    Stir the tomatoes and soup together then pour over steaks and veggies.\par
    Add a little water or beef broth.\par
    Cover with foil and cook at 350 degrees about an hour or until veggies are soft.\par
    \emph{Note: Boss likes to brown the steak in a cast iron skillet then transfer it to the oven.}
\end{method}

% Chicken & Dumplings
\recipe[Perfect for those cold dreadful days.]{Chicken \& Dumplings}{Shug}
\serves{4}
\preptime{10 minutes}
\cooktime{8--9 hours}
\dishtype{}
\begin{ingreds}
    3--4 chicken breasts
    1 medium onion (diced)
    2 stalks of celery (chopped)
    3--4 carrots (sliced or diced)
    2 cans cream of chicken soup
    1 box chicken broth
    3 tbsp butter
    garlic powder
    parsley
    poultry seasoning
    black pepper
    1 cup half \& half
    1 can Pillsbury\textsuperscript{\textregistered} buttermilk biscuits
\end{ingreds}
\begin{method}
    Place chicken in the crockpot.\par
    Stir together all other ingredients except for the half \& half and biscuits and pour over the chicken.\par
    Cook on low for 7--8 hours.\par
    Take out the chicken and pull apart.\par
    Cut the raw biscuits into small $\frac{3}{4}$'' pieces and place in the crockpot.\par
    Add the half \& half.`\ '\emph{You can add how ever much you want.}\par
    Put the chicken back in and cook on high for another hour.
\end{method}

%%%%%%%%%%%%%%%%%%%%%%%%%%%%%%%%%%%%%%%%%%%%%%%%%%%%%%%%%%%%%%

\recipeSection{Desserts}

%%%%%%%%%%%%%%%%%%%%%%%%%%%%%%%%%%%%%%%%%%%%%%%%%%%%%%%%%%%%%%

% Creme Brulee
\recipe[There's something wrong with you if you don't like this dish.]{Cr\'eme Br\^ul\'ee}{Gage}
\serves{4}
\preptime{20 minutes}
\cooktime{30--60 minutes}
\dishtype{}
\begin{ingreds}
    2 cups heavy cream
    5 egg yolks
    $\frac{1}{2}$ cup granulated sugar
    1 tsp vanilla extract
    $\frac{1}{4}$ tsp kosher salt
    sugar for topping
    Fruit of choice for topping
\end{ingreds}
\begin{method}[Preheat oven to 325 degrees. A torch will work better than a broiler for caramelizing the sugar topping.]
    In a saucepan, combine cream, vanilla bean, and salt, and cook over low heat just until beginning to steam. Let sit for 5--10 minutes.\par
	In a bowl, beat yolks and sugar together until light.\par
    Stir about a quarter of the cream into this mixture, then pour sugar-egg mixture into cream and stir.\par
    Pour into four 6-ounce ramekins and place ramekins in a baking dish\par
    Fill dish with water halfway up the sides of the dishes.\par
    Bake for 30 min to 1 hour, or until centers are barely set.\par
    Cool completely. Refrigerate for several hours or up to two days.\par
	When ready to serve, top each custard with about a teaspoon of sugar in a thin layer.\par
    Place ramekins in a broiler 2--3'' from heat source.\par
    Turn on the broiler and cook until the sugar melts and browns or even blackens a bit, about 5 minutes.\par
    Serve within two hours.
\end{method}

% Raspberry Lemon Ice Cream
\recipe[Try it with a chocolate sauce. Trust me.]{Raspberry Lemon Ice Cream}{Boss}
\serves{4}
\preptime{5 minutes}
\cooktime{1--2 hours}
\dishtype{}
\begin{ingreds}
    1 can Eagle Brand\textsuperscript{\textregistered} sweetened condensed milk
    2 eggs
    2 cups half \& half
    1 small lemon (juice and zest)
    1 tsp vanilla
    $\frac{1}{8}$ tsp almond extract
    1 cup fresh raspberries
\end{ingreds}
\begin{method}
    Put the raspberries in the freezer (they don’t have to be frozen, just cold).\par
    Mix together all the ingredients, excluding the raspberries.\par
    Pour mixture into your ice cream maker and turn it on.\par
    About halfway through, when the ice cream is starting to thicken, add the raspberries.\par
    \emph{Note: also works with strawberries.}
\end{method}

% Sopapilla Cheesecake
\recipe[The most addicting thing ever devised.]{Sopapilla Cheesecake}{Shug}
\serves{4}
\preptime{30 minutes}
\cooktime{1 hour}
\dishtype{}
\begin{ingreds}
    2 packages cream cheese (softened)
    1$\frac{1}{4}$ sticks butter (softened, divided)
    1$\frac{1}{2}$ cups sugar (divided)
    2 tsp vanilla (divided)
    2 packages Pillsbury\textsuperscript{\textregistered} crescent rolls 
    $\frac{1}{2}$ tsp ground cinnamon
\end{ingreds}
\begin{method}[Preheat the oven to \temp{350}.]
    Grease a 13$\times$9'' pan and set aside.\par
    Combine cream cheese and 4 tbsp butter in large bowl.\par
    Beat until mixture is creamy.\par
    Gradually beat in 1 cup of sugar and 1 tsp vanilla.\par
    Unroll the package of crescent rolls and lay in the prepared pan (do not separate the rolls).\par
    Stretch the dough to the edges of the pan and seal the perforations.\par
    Spread the cream cheese mixture evenly over the dough.\par
    Place the remaining package of crescent rolls on the top of the cream cheese mixture.\par
    Pinch the dough together at the perforations to seal.\par
    Melt the remaining 6 tbsp butter.\par
    Stir in the remaining sugar and vanilla as well as the cinnamon.\par
    Pour mixture evenly over dough.\par
    Bake for 30 minutes or until the cheesecake is slightly puffed and golden brown.
\end{method}

% Frosted Brownies
\recipe[Ok, maybe this is the most addicting thing ever devised?]{Frosted Brownies}{Mimi}
\serves{4}
\preptime{20 minutes}
\cooktime{1 hour}
\dishtype{}
\begin{ingreds}
    \ingredients[Brownie Mix]
        2 cups sugar
        1$\frac{1}{2}$ cups flour
        $\frac{1}{3}$ cup cocoa 
        1 tsp salt
        4 eggs
        2 sticks butter
        2 tsp vanilla
    \columnbreak{}
    \ingredients[Icing]
        1 stick butter
        $\frac{1}{4}$ cup cocoa
        3$\frac{1}{2}$ cups powdered sugar
        $\frac{1}{4}$ cup half \& half
\end{ingreds}
\begin{method}[Preheat the oven to \temp{350}.]
    Grease and flour 9$\times$13'' dish and set aside.\par
    Add the sugar, flour, and salt to a large mixing bowl.\par
    Melt the butter and cocoa in the microwave.\par
    While mixing, add the eggs to the dry ingredients. Once incorporated, add the butter/cocoa mixture and the vanilla.\par
    Bake for 30 minutes. Let cool completely.\par
    For the icing, melt the butter and cocoa.\par
    Add the half \& half.\par
    Sift in the powdered sugar. Add more half \& half if needed.\par
    Layer the icing on top of the cooled brownies.    
\end{method}

% Browned Butter Brownies
\recipe[It has coffee, so eat it for the energy.]{Browned Butter Brownies}{Gage}
\serves{4}
\preptime{30 minutes}
\cooktime{30 minutes}
\dishtype{}
\begin{ingreds}
    1$\frac{1}{2}$ sticks unsalted butter
    4 oz (113 grams) semisweet chocolate (chopped)
    $\frac{1}{2}$ cup (100 grams) sugar
    $\frac{1}{2}$ cup (100 grams) light brown sugar
    3 large eggs
    1 tsp vanilla extract
    $\frac{1}{2}$ cup (64 grams) all-purpose flour
    $\frac{1}{2}$ cup (43 grams) unsweetened cocoa powder
    $\frac{1}{2}$ teaspoon instant espresso powder
    $\frac{1}{2}$ teaspoon fine salt
    1 cup (170 grams) semisweet chocolate chips
\end{ingreds}
\begin{method}[Preheat the oven to \temp{350}.]
    Grease a glass 8'' square pan.
    In a small saucepan set over medium low heat, melt the butter.\par
    Swirling the pan occasionally, continue to cook the butter, increasing the heat to medium. It should become foamy with audible crackling and popping noises.\par
    Once the crackling stops, continue to swirl the pan until the butter develops a nutty aroma and brown bits start to form at the bottom. Do not allow the bits to burn!\par
    Once the bits are amber in color, ($\sim$2--3 minutes after the popping stops) remove from heat and pour into a mixing bowl, scraping the brown bits into the bowl.\par
    Immediately add in the chopped chocolate and stir until melted.\par
    In the bowl of an electric mixer fitted with the whisk attachment, combine the sugar, brown sugar, eggs, and vanilla.\par
    Beat on high speed until completely thickened and fluffy in texture and lightened in color, $\sim$8 minutes.\par
    On low speed, gradually pour in the butter/chocolate mixture.\par
    Use a rubber spatula to fold in the flour, cocoa powder, espresso powder, and salt until just combined.\par
    Fold in the chocolate chips.\par
    Pour into the prepared pan and smooth out with a spatula.\par
    Bake for 22--25 minutes, or until set but not overbaked.\par
    Let cool completely before slicing and serving.       
\end{method}

% Vanilla Pudding
\recipe[It's vanilla pudding!]{Vanilla Pudding}{Shug}
\serves{4}
\preptime{10 minutes}
\cooktime{15 minutes}
\dishtype{}\begin{ingreds}
    $\frac{1}{2}$ cup sugar
    2 tbsp cornstarch
    $\frac{1}{2}$ tsp salt
    2 cups milk
    2 large egg yolks (slightly beaten)
    2 tbsp butter (softened)
    2 tsp vanilla
\end{ingreds}
\begin{method}
    In a 2-quart saucepan, mix the sugar, cornstarch, and salt.\par
    Gradually stir in the milk.\par
    Cook over medium heat, stirring constantly, until mixture thickens and boils.\par
    Boil and stir for 1 minute.\par
    Gradually stir $\geq\frac{1}{2}$ of the hot mixture into the egg yolks to temper them, then stir back into hot mixture in saucepan.\par
    Boil and stir for another minute.\par
    Remove from heat and stir in the butter and vanilla.       
\end{method}

\end{document}