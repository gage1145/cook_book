\documentclass{article}
\usepackage{fancyhdr}
\usepackage{multicol}
\usepackage[
    % a5paper,
    papersize={5.5in,8.5in},
    margin=0.75in,
    top=0.75in,
    bottom=0.75in,
    twoside
    ]{geometry}
\usepackage{xcolor}
\usepackage{graphicx}

\setlength{\headheight}{14.0pt}

\raggedcolumns{}
\setlength{\multicolsep}{0pt}
\setlength{\columnseprule}{1pt}

\makeatletter

% Decide if you want numbered sections or not.
\setcounter{secnumdepth}{0}

% Used for the headnote and in \showit
% If the text is small it is placed on one line;
% otherwise it is put into a raggedright paragraph.
\long\def\testoneline#1{%
  \sbox\@tempboxa{#1}%
  \ifdim\wd\@tempboxa<0.75 \linewidth{}
        \begingroup
            \itshape{}
            #1\par
        \endgroup
  \else
    \parbox{0.75\linewidth}{\raggedright\itshape#1}
    \par
  \fi
}

\newif\if@mainmatter{} \@mainmattertrue{}

% Borrowed from book.cls
\newcommand\frontmatter{%
    \cleardoublepage{}
  \@mainmatterfalse{}
  \pagenumbering{roman}}
\newcommand\mainmatter{%
    \cleardoublepage{}
  \@mainmattertrue{}
  \pagenumbering{arabic}}
\makeatother

% Vary the colors at will
\definecolor{vegcolor}{rgb}{0,0.5,0.2}
\definecolor{frzcolor}{rgb}{0,0,1}
\definecolor{dessertcolor}{rgb}{0.5,0.2,0.1}
\definecolor{makeaheadcolor}{rgb}{0.5,0.5,0.6}
\definecolor{breadcolor}{rgb}{0.9, 0.5, 0.2}
\definecolor{appcolor}{rgb}{0.2, 0.5, 1.0}
\definecolor{saucecolor}{rgb}{0.2, 0.5, 0.1}

%% Thanks to alephzero for the excellent start:
\newcommand{\recipe}[3][]{
    \newpage
    \thispagestyle{fancy}
    \lhead{}
    \chead{}
    \rhead{}
    \lfoot{}
    \rfoot{}
    \subsection[#2]{\LARGE#2 $-$ \large#3}
    \if#1
    \else
        \begin{center}
            \testoneline{#1}
        \end{center}
    \fi
}
\newcommand{\serves}[1]{\chead{Serves #1}}
\newcommand{\dishtype}[1]{\rhead{#1}}
\newcommand{\dishother}[1]{\lhead{#1}}
\newcommand{\vegetarian}{\large\color{vegcolor}\textbf{V}}
\newcommand{\bread}{\large\color{breadcolor}\textbf{B}}
\newcommand{\apps}{\large\color{appcolor}\textbf{A}}
\newcommand{\sauce}{\large\color{saucecolor}\textbf{S}}
\newcommand{\freeze}{\large\color{frzcolor}\textbf{F}}
\newcommand{\dessert}{\large\color{dessertcolor}\textbf{D}}
\newcommand{\makeahead}{\large\color{makeaheadcolor}\textbf{M}}

% Optional arguments for alternate names for these:
\newcommand{\preptime}[2][Prep time]{\lfoot{#1: #2}}
\newcommand{\cooktime}[2][Cook time]{\rfoot{#1: #2}}
\newcommand{\temp}[1]{$#1^\circ$F}

%% Optional argument is the width of the graphic, default = 1in
\newcommand{\showit}[3][1in]{
    \begin{center}
        \bigskip
            \includegraphics[width=#1]{#2}
            \par
            \medskip
            \testoneline{#3}
            \par
    \end{center}
}

%% Optional argument for a  heading within the ingredients section
\newcommand{\ingredients}[1][]{%
    \if###1##{
        \color{red}\Large\textbf{Ingredients}
    }
    \else
        \emph{#1}
    \fi
}

\newcommand{\recipeSection}[1]{
    \clearpage
    \begin{center}
        \hspace{0pt}\vfill
        \begin{minipage}{\textwidth}
            \centering
            \section[#1]{\huge#1}
        \end{minipage}
        \vfill\hspace{0pt}
    \end{center}
    \clearpage
}

%% Use \obeylines to minimize markup
\newenvironment{ingreds}{
    \parindent0pt
    \noindent
    \ingredients{}
    \par
    \smallskip
    \begin{multicols}{2}
    \leftskip1em
    \rightskip0pt plus 3em
    \parskip=0.25em
    \obeylines{}
    \everypar={\hangindent2em}
}{
    \end{multicols}
    \medskip
}

\newcounter{stepnum}

%% Optional argument for an italicized pre-step
%% Also use obeylines to minimize markup here as well
\newenvironment{method}[1][]{%
    \setcounter{stepnum}{0}
    \noindent
    {\color{red}\Large\textbf{Instructions}}
    \par
    \smallskip
    \if#1
    \else
        \noindent
        \emph{#1}
        \par
    \fi
    \begingroup
    \parindent0pt
    \parskip0.25em
        \leftskip2em
    \everypar={\llap{\ensuremath{\stepcounter{stepnum}\hbox to2em{\thestepnum.\hfill}}}}
}{%
    \par
    \endgroup
}

\pagestyle{plain}

\title{Rowden Family Cookbook}
\author{Gage R. Rowden}

\begin{document}

\maketitle

\frontmatter
\tableofcontents

\mainmatter{}

%%%%%%%%%%%%%%%%%%%%%%%%%%%%%%%%%%%%%%%%%%%%%%%%%%%%%%%%%%%%%%

\recipeSection{Breads}

%%%%%%%%%%%%%%%%%%%%%%%%%%%%%%%%%%%%%%%%%%%%%%%%%%%%%%%%%%%%%%

\recipe[Perfect for making Cubano sandwiches.]{Cubano Bread}{Gage}
\serves{4}
\preptime{2 hour}
\cooktime[Cook time]{28--35 minutes}
\dishtype{\vegetarian, \bread}
\dishother{\makeahead}
\begin{ingreds}
    1$\frac{1}{4}$ cups water (\temp{100})
    1 tbsp instant dry yeast
    3$\frac{1}{2}$ cups (500g) unbleached bread flour
    1 tbsp (13g) granulated sugar
    2 tsp (12g) fine sea salt
    2$\frac{1}{2}$ tbsp lard, solidified
% \columnbreak
% \ingredients[For the Crumble Mixture:]
%      80g wholemeal flour
%      80g plain flour
%      80g butter (diced)
%      70g demerara sugar
\end{ingreds}
\begin{method}[Preheat the oven to \temp{400}.]
    In a small bowl, stir together the warm water and yeast. Cover with plastic wrap and let sit for 10 min. The mixture should get slightly foamy, and the yeast should dissolve.\par
    In a stand mixer bowl, add the flour, sugar, and salt. Mix together thoroughly. With the dough hook attachment, start mixing on medium-low speed. Slowly add the warm yeast mixture and the lard. Mix until combined. Once a cohesive dough is formed, keep mixing for another 3 to 5 minutes, or until smooth.\par
    Shape the dough into a ball, and place in a greased bowl covered with greased plastic wrap. Let rise at room temp for 45 min to 1 hr, or until doubled in size.\par
    Punch down the dough to release the gas and place on a lightly floured surface. Divide the dough into 2 even pieces. Cover with a damp towel, and let them rest right where they are for 10 minutes.\par
    Flatten out 1 piece of dough into about a $\frac{1}{2}$-inch thick rectangle, with the long edge $\sim$10'' long.\par
    From the long edge, tightly roll the dough and close the seams at the bottom and sides.\par
    Carefully roll the log while applying pressure outward to \linebreak slightly taper the ends. The log should be $\sim$15'' long.\par
    Repeat with the other piece of dough.\par
    Place the loaves onto a parchment-lined baking sheet 4--6'' apart. Cover with another baking sheet.\par
    Let the dough proof for $>$30 minutes at room temperature.\par
    About 10 minutes before the dough is done proofing, bring a 10'' oven-proof skillet of water to a boil.\par
    Using a spray bottle, lightly spray the dough with water.\par
    Using a razor blade or an extremely sharp knife, score a shallow seam along the length of the loaves.\par
    Place the skillet of boiling water on the bottom rack of the oven.\par
    Place the baking sheet with the dough in the middle of the oven on a separate rack above the water.\par
    Spray the inside of the oven with a little water to generate steam.\par
    Let the bread steam for 8--10 minutes. Reduce the heat to \temp{375}, remove the skillet of water, and let the bread bake for another 20--22 minutes, or until lightly browned.\par
    Let cool completely on a wire rack. Slice and serve. Store on the counter loosely wrapped in a kitchen towel for up to 2 days and then in a resealable bag for 1 more day. Freeze in an airtight container for up to 2 months.\par
\end{method}
% \showit[1.25in]{example-image-b}{This is a picture}

% Browned Butter Cornbread
\recipe[Delicious and savory.]{Browned Butter Cornbread}{Gage}
\serves{4}
\preptime{1 hour}
\cooktime[Cook time]{25 minutes}
\dishtype{\vegetarian, \bread}
\dishother{\makeahead}
\begin{ingreds}
    $\frac{1}{2}$ cup (112 g) unsalted butter
    $\frac{1}{2}$ bunch of sage
    $\frac{1}{2}$ bunch of thyme
    1$\frac{1}{4}$ cups (188 g) all-purpose flour
    $\frac{1}{3}$ cup (67 g) sugar
    3$\frac{1}{2}$ tbsp (47 g) brown sugar
    1$\frac{1}{4}$ tsp (8 g) kosher salt
    1 tbsp (13 g) baking powder
    1$\frac{1}{4}$ cups (175g) finely ground yellow cornmeal
    2 large eggs
    1$\frac{1}{2}$ cups buttermilk
\end{ingreds}
\begin{method}[Preheat an 8” cast-iron skillet in the oven at \temp{400}]
    In a small pot, melt the butter over medium heat. Once melted, continue cooking to brown the butter until it turns dark brown. Then turn off the heat.\par
    Add the sage and thyme, stir to combine, transfer the butter-herb mixture to a separate container, and set aside to cool.\par
    In a bowl, whisk together the flour, sugar, brown sugar, salt, baking powder, and cornmeal.\par
    In a separate bowl, whisk the eggs, then whisk in the buttermilk.
    Discard the herbs from the slightly cooled butter. In a slow stream, whisk the butter into the buttermilk mixture. Then whisk the wet mixture into the dry mixture until completely smooth and combined.\par
    Remove the hot pan from the oven, and lightly grease it with cooking spray. Add the batter to the pan, and spread it evenly.\par
    Bake for 20--25 min, or until a toothpick inserted into the center comes out clean. 
    Cool for a few minutes in the pan and then remove to a cooling rack to cool completely.\par
\end{method}

% Banana Bread
\recipe[This will make anyone's day batter.]{Banana Bread}{Shug}
\serves{4}
\preptime{5 minutes}
\cooktime{15 minutes}
\dishtype{\vegetarian, \sauce}
\begin{ingreds}
    1 red bell pepper
    1 green bell pepper
    10 jalape\~nos
    1$\frac{1}{2}$ cups white vinegar
    $\frac{1}{2}$ tsp salt
    6 cups sugar
    1 pouch liquid fruit pectin   
\end{ingreds}
\begin{method}
    In food processor, finely chop the peppers.\par
    Place peppers in large pot with vinegar, salt, and sugar.\par
    Boil for 10 minutes, stirring often (be careful…it will boil over QUICK!).\par
    Add pectin pouch and boil 1 more minute.\par
    Fill canning jars.\par
    Add to boiling hot water bath for 10 minutes.\par
    Wait a day or so for jelly to set.
\end{method}


%%%%%%%%%%%%%%%%%%%%%%%%%%%%%%%%%%%%%%%%%%%%%%%%%%%%%%%%%%%%%%

\recipeSection{Appetizers \& Snacks}

%%%%%%%%%%%%%%%%%%%%%%%%%%%%%%%%%%%%%%%%%%%%%%%%%%%%%%%%%%%%%%

\recipe[Made at every family get-together.]{Guacamole}{Boss}
\serves{4}
\preptime{15 minutes}
\cooktime[Cook time]{0 minutes}
\dishtype{\vegetarian, \apps}
\begin{ingreds}
    4 Avocados
    $\frac{1}{3}$ onion
    $\frac{1}{2}$ jalapeno
    $\frac{1}{4}$ cup cilantro
    1 tbsp mayo
    2 tsp vinegar
    $\frac{1}{2}$ lemon
    garlic salt
    pepper
    tomatoes (optional)
    stuffed olives (optional)
    cucumbers (optional)
\end{ingreds}
\begin{method}
    Mix and mash all the ingredients together.
\end{method}

%%%%%%%%%%%%%%%%%%%%%%%%%%%%%%%%%%%%%%%%%%%%%%%%%%%%%%%%%%%%%%

\recipeSection{Sauces, Jams, \& Canned Goods}

%%%%%%%%%%%%%%%%%%%%%%%%%%%%%%%%%%%%%%%%%%%%%%%%%%%%%%%%%%%%%%

% Tex-Mex Enchilada Gravy
\recipe[White people can make it too!]{Tex-Mex Enchilada Gravy}{Gage}
\serves{4}
\preptime{5 minutes}
\cooktime{15 minutes}
\dishtype{\vegetarian, \sauce}
\begin{ingreds}
    $\frac{1}{4}$ cup oil or butter
    $\frac{1}{4}$ cup all purpose flour
    $\frac{1}{2}$ tsp ground black pepper
    $\frac{3}{4}$ tsp salt
    2 tsp garlic powder
    2 tsp ground cumin
    $\frac{1}{2}$ tsp dried oregano
    1 tbsp chili powder
    2 cups beef broth
    1 tbsp tomato paste
\end{ingreds}
\begin{method}
    Heat the oil or butter in a medium size skillet over medium heat.
    Add in the flour and stir to mix. Will be a thick mixture.\par
    Allow this to cook for just 1--2 minutes. The roux should be a very light brown.
    Add in the spices. Stir to form a thick paste for no longer than 30 seconds.
    Stir in the broth and tomato paste until smooth.\par
    Continue to heat while stirring for several minutes until the sauce is slightly thickened.\par
    Keep the heat to just below a simmer.\par
    Pour into a jar to store or use to make enchiladas.
\end{method}

% Boss' Tartar Sauce
\recipe[Enjoy with any fish or just by the spoonful.]{Boss' Tartar Sauce}{Boss}
\serves{4}
\preptime{5 minutes}
% \cooktime{0 minutes}
\dishtype{\vegetarian, \sauce}
\begin{ingreds}
    1 cup real mayo
    $\frac{1}{2}$ cup onion (finely chopped)
    2 tsp dill relish
    2 tsp sweet relish
    $\frac{1}{2}$ jalapeno 
    $\frac{1}{4}$ tsp thyme
    $\frac{1}{2}$ lemon (juice)
    Salt
    Pepper
\end{ingreds}
\begin{method}
    Mix all ingredients together in a bowl.
    Season to taste.
\end{method}

% Katsu Sauce
\recipe[Sweet, tangy, and savory. Perfect for dipping fried food!]{Katsu Sauce}{Gage}
\serves{4}
\preptime{5 minutes}
% \cooktime{0 minutes}
\dishtype{\vegetarian, \sauce}
\begin{ingreds}
    6 tbsp ketchup
    2 tbsp Worcestershire sauce
    2 tbsp oyster sauce
    2 tbsp miso paste (optional)
    2 tsp honey
\end{ingreds}
\begin{method}
    In a small bowl, stir together all of the ingredients until fully incorporated.\par
    Use or store in an airtight container in the refrigerator for up to 1 month.
\end{method}

% Plum Preserves
\recipe[For when your plum tired of life.]{Plum Preserves}{Shug}
\serves{4}
\preptime{5 minutes}
\cooktime{30 minutes}
\dishtype{\vegetarian, \sauce}
\begin{ingreds}
    8 cups plums (pitted/cut)
    4 cups sugar
\end{ingreds}
\begin{method}
    Put plums and sugar in the instant pot.\par
    Saut\'e for on medium-high for 3 minutes.\par
    Pressure cook on high for 1 minute.\par
    Slow release for 10 minutes, then quick release.\par
    Mash and stir on the saut\'e setting until thick but pourable (stirring every 5 minutes).
\end{method}

% Apricot Preserves
\recipe[Perfect pairing for white bread and peanut butter.]{Apricot Preserves}{Shug}
\serves{4}
\preptime{5 minutes}
\cooktime{30 minutes}
\dishtype{\vegetarian, \sauce}
\begin{ingreds}
    8 cups apricots (pitted/cut)
    6 cups sugar
    ¼ cup lemon juice
\end{ingreds}
\begin{method}
    Put apricots and sugar in the instant pot.\par
    Saut\'e on medium-high for 3 minutes.\par
    Pressure cook on high for 1 minute.\par
    Slow release for 10 minutes, then quick release.\par
    Mash and stir on the saut\'e setting until thick but pourable (stirring every 5 minutes).
\end{method}

% Raspberry Jam
\recipe[Try it with chocolate sauce and ice cream.]{Raspberry Jam}{Shug}
\serves{4}
\preptime{5 minutes}
\cooktime{30 minutes}
\dishtype{\vegetarian, \sauce}
\begin{ingreds}
    5 cups raspberries (4 small containers)
    2 cups sugar (more or less to taste)
    Juice from 2 large lemons
    2 tbsp cornstarch
    2 tbsp water    
\end{ingreds}
\begin{method}
    Add the raspberries, sugar, and lemon juice to the instant pot and mix.\par
    Pressure cook on high for 3 minutes.\par
    Slow release for 10 minutes, then quick release. Saut\'e and stir until you reach your desired consistency.\par
    While releasing the pressure, make a slurry with the cornstarch and water.\par
    Add the slurry in small portions to the mixture while stirring until desired viscosity is reached.
\end{method}

% Jalapeno Jam
\recipe[Delicious with cream cheese and crackers!]{Jalape\~no Jelly}{Shug}
\serves{4}
\preptime{5 minutes}
\cooktime{15 minutes}
\dishtype{\vegetarian, \sauce}
\begin{ingreds}
    1 red bell pepper
    1 green bell pepper
    10 jalape\~nos
    1$\frac{1}{2}$ cups white vinegar
    $\frac{1}{2}$ tsp salt
    6 cups sugar
    1 pouch liquid fruit pectin   
\end{ingreds}
\begin{method}
    In food processor, finely chop the peppers.\par
    Place peppers in large pot with vinegar, salt, and sugar.\par
    Boil for 10 minutes, stirring often (be careful…it will boil over QUICK!).\par
    Add pectin pouch and boil 1 more minute.\par
    Fill canning jars.\par
    Add to boiling hot water bath for 10 minutes.\par
    Wait a day or so for jelly to set.
\end{method}

%%%%%%%%%%%%%%%%%%%%%%%%%%%%%%%%%%%%%%%%%%%%%%%%%%%%%%%%%%%%%%

\recipeSection{Breakfast}

%%%%%%%%%%%%%%%%%%%%%%%%%%%%%%%%%%%%%%%%%%%%%%%%%%%%%%%%%%%%%%

% Buttermilk Pancakes
\recipe[Make these every Christmas morning.]{Buttermilk Pancakes}{Shug}
\serves{4}
\preptime{5 minutes}
\cooktime{15 minutes}
\dishtype{\vegetarian}
\begin{ingreds}
    2 eggs, separated
    2 cups buttermilk
    2 cups flour
    $\frac{1}{4}$ tsp salt
    $\frac{1}{2}$ cup sugar
    3 tsp baking powder
    $\frac{1}{4}$ tsp baking soda
    2 tbsp vegetable oil
    1 tsp vanilla 
\end{ingreds}
\begin{method}
    Separate eggs, beat egg whites until stiff, and set aside.\par
    Mix egg yolks with buttermilk.\par
    Mix dry ingredients and oil.\par
    Add egg yolks and buttermilk to the dry mixture. Mix well.\par
    Stir in vanilla.\par
    Fold in the egg whites.\par
    Cook on medium-high griddle.
\end{method}

%%%%%%%%%%%%%%%%%%%%%%%%%%%%%%%%%%%%%%%%%%%%%%%%%%%%%%%%%%%%%%

\recipeSection{Side Dishes}

%%%%%%%%%%%%%%%%%%%%%%%%%%%%%%%%%%%%%%%%%%%%%%%%%%%%%%%%%%%%%%

% Thanksgiving Dressing
\recipe[Throwback to the OG Shug.]{Thanksgiving Dressing}{Great Shug}
\serves{4}
\preptime{15 minutes}
\cooktime{45 minutes}
\dishtype{\vegetarian}
\begin{ingreds}
    1 cup chopped onion
    1 cup chopped celery
    1 stick butter (melted)
    2--3 packages baked cornbread
    1 can baked biscuits
    1 container chicken broth
    1 can cream of celery, mushroom, or chicken soup 
\end{ingreds}
\begin{method}[Preheat oven to \temp{375}.]
    Mix cornbread, biscuits, onion, celery, and butter together.\par
    Add chicken broth and soup to get the consistency you want.\par
    Bake until golden brown on top (use the broiler if desired).
\end{method}

% Carrots & Kale
\recipe[For the rabbits in the family.]{Carrots \& Kale}{Shug}
\serves{4}
\preptime{15 minutes}
\cooktime{15 minutes}
\dishtype{\vegetarian}
\begin{ingreds}
    3 large carrots (sliced)
    1 bushel of kale (chopped)
    1 onion (sliced)
    2 tsp garlic salt
    1 tsp lemon pepper
    1 tbsp olive oil
    1 tbsp butter    
\end{ingreds}
\begin{method}
    Sear carrots in olive oil and butter for about 5 minutes.\par
    Add the onion and continue searing with the lid on the pan, without stirring, until carrots are dark and carmelized.\par
    Stir, then add chopped kale, turn off the burner, and cover until kale is wilted.\par
    Add the seasonings to taste.
\end{method}

% Brussels Sprouts
\recipe[These will give lasting luck if consumed on New Year's Eve.]{Brussels Sprouts}{Boss}
\serves{4}
\preptime{15 minutes}
\cooktime{30 minutes}
\dishtype{\vegetarian}
\begin{ingreds}
    2 lbs Brussels sprouts (halved or quartered)
    1 onion (sliced) 
    1 tbsp olive oil
    1 tbsp butter
    2 tsp garlic salt
    1 tsp lemon pepper        
\end{ingreds}
\begin{method}
    Place brussel sprouts flat side down in pan of olive oil and butter.\par
    Add the onion.\par
    Season with garlic salt and lemon pepper.\par
    Cover and sear until brussel sprouts are dark and carmelized.
\end{method}

% Brussels Sprouts
\recipe[For when you want to get slawpy.]{Coleslaw}{Boss}
\serves{4}
\preptime{15 minutes}
% \cooktime{30 minutes}
\dishtype{\vegetarian}
\begin{ingreds}
    1 small head red cabbage
    4 carrots (thinly sliced)
    1 jalapeño (finely diced)
    4 green onions (cut in ribbons)
    1 bunch cilantro (roughly chopped)
    1 cup REAL mayo
    1 tsp cumin
    1 tsp smoked paprika
    2 tsp garlic salt
    black pepper (to taste)
    1 tsp vinegar       
\end{ingreds}
\begin{method}
    Mix all the ingredients together in a bowl.
\end{method}

% White Rice
\recipe[Simple and goes with literally everything.]{White Rice}{Gage}
\serves{4}
\preptime{5 minutes}
\cooktime{25 minutes}
\dishtype{\vegetarian}
\begin{ingreds}
    1 cup white rice
    2 cups water
    1 tsp Kosher salt          
\end{ingreds}
\begin{method}
    In a fine mesh strainer add the rice.\par
    Place the strainer into a bowl, and fill with water.\par
    Mix the rice in the water and dispose of the water.\par
    Repeat this process until the water no longer becomes cloudy.\par
    Add the rice to a medium sauce-pot along with 2 cups of room temperature water and salt.\par
    On the smallest burner, bring to a boil, then reduce the heat to low and cover.\par Simmer for 20 minutes.
\end{method}

%%%%%%%%%%%%%%%%%%%%%%%%%%%%%%%%%%%%%%%%%%%%%%%%%%%%%%%%%%%%%%

\recipeSection{Mains}

%%%%%%%%%%%%%%%%%%%%%%%%%%%%%%%%%%%%%%%%%%%%%%%%%%%%%%%%%%%%%%

% Fettuccine Alfredo
\recipe[It slaps.]{Fettuccine Alfredo}{Jacy}
\serves{4}
\preptime{5 minutes}
\cooktime{25 minutes}
\dishtype{}
\begin{ingreds}
    $\frac{1}{2}$ cup butter
    1 $\frac{1}{2}$ cup heavy whipping cream
    2 tsp minced garlic
    $\frac{1}{2}$ tsp italian seasoning
    $\frac{1}{2}$ tsp salt
    $\frac{1}{4}$ tsp pepper
    2 cups grated parmesan cheese
    1 bag fettuccine noodles
    1 lb chicken breast or shrimp
\end{ingreds}
\begin{method}
    Add butter and cream to a large skillet.\par
    Simmer over low for 2 minutes.\par
    Whisk in garlic and seasonings for one minute.\par
    Whisk in parmesan until melted.\par
    Mix in with noodles and your choice of chicken or shrimp.\par
    Simmer until the meat is cooked.
\end{method}

% Chili
\recipe[Make this every fall.]{Chili}{Boss}
\serves{4}
\preptime{5 minutes}
\cooktime{2--4 hours}
\dishtype{}
\begin{ingreds}
    1 lb ground beef
    1 onion
    3 garlic cloves
    2 jalapenos 
    1 can stewed tomatoes (16 oz.)
    1 can tomato sauce (16 oz.)
    3 tbsp chili powder (at least)
    1 tbsp pepper
    1 tbsp cumin
    1 tbsp garlic salt
    1 tsp paprika
    $\frac{1}{4}$ tsp coriander
    2 tsp rosemary
    2 cups water
\end{ingreds}
\begin{method}
    Saute onion, garlic, and jalapenos with olive oil.\par
    In a separate pan, brown the meat.\par
    Once browned, drain the grease, and add to the sauteed vegetables.\par
    Add all other ingredients.\par
    Simmer for 2--4 hours.\par
    Seasoning does not have to be exact, just add to your liking. 
\end{method}

% Chili 2.0
\recipe[Gage's rendition on Boss' chili.]{Chili 2.0}{Gage}
\serves{4}
\preptime{5 minutes}
\cooktime{2--4 hours}
\dishtype{}
\begin{ingreds}
    1 tbsp peanut oil
    1 lb ground beef
    1 lb breakfast sausage
    4 strips of bacon sliced 
    1 onion finely diced
    3 garlic cloves crushed
    3 jalape\~nos finely diced
    $\frac{1}{4}$ cup flour
    $\frac{1}{3}$ cup red wine (optional)
    1 large can crushed tomatoes
    $\geq$3 tbsp chili powder
    1 tbsp pepper
    1 tbsp cumin
    1 tbsp garlic powder
    1 tsp paprika
    $\frac{1}{4}$ tsp coriander
    2 tsp rosemary
    Salt (to taste)
    2 cups water
\end{ingreds}
\begin{method}
    In a large pot, add the oil and heat over high until shimmering.\par
    Add the bacon and cook until lightly browned.\par
    Add the beef and breakfast sausage.\par
    Cook over high until browned.\par
    Remove the meat and add the onions and jalape\~nos.\par
    Cook until translucent.\par
    Add the flour and thoroughly mix.\par
    Cook until lightly browned.\par
    Add the garlic, and continue to mix for another 30 sec.\par
    Deglaze with the wine or beef stock, then add the crushed tomatoes and all the spices.\par
    Mix thoroughly, and add water to desired thickness.\par
    Cover, reduce heat to low, and simmer for 2--4 hours. 
\end{method}

% Blackened Catfish
\recipe[Simple, easy, delicious.]{Blackened Catfish}{Boss}
\serves{4}
\preptime{5 minutes}
\cooktime{10 minutes}
\dishtype{}
\begin{ingreds}
    2 catfish filets
    1$\frac{1}{2}$ tsp garlic salt
    1$\frac{1}{2}$ tsp paprika  
    black pepper (to taste)  
\end{ingreds}
\begin{method}
    Season fish to your liking on both sides. Use pepper generously.\par
    Cook on medium-high heat in olive oil for approximately 3 minutes on each side, or until done. Cast iron skillet will produce the best results. 
\end{method}

% Tomato Soup
\recipe[Of course you have to eat it with a grilled cheese!]{Tomato Soup}{Shug}
\serves{4}
\preptime{10 minutes}
\cooktime{30 minutes}
\dishtype{}
\begin{ingreds}
    $\frac{1}{2}$ onion
    6 tbsp butter
    1 can diced tomatoes
    1 can tomato sauce
    $\frac{1}{4}$ cup chicken broth
    3 tbsp sugar
    1 cup half \& half
    salt and pepper (to taste)
    1$\frac{1}{2}$ tsp parsley
    1$\frac{1}{2}$ tsp basil    
\end{ingreds}
\begin{method}
    Saut\'e onions in the butter.\par
    Process the sauteed onions and diced tomatoes in a food processor.\par
    Put back in the pot to mix with tomato sauce and chicken broth.\par
    Bring soup to almost a simmer.\par
    Add sugar, spices, and half \& half.
\end{method}

% Tomato Bisque
\recipe[If you liked the last one, you'll love this!]{Tomato Bisque}{Gage}
\serves{4}
\preptime{20 minutes}
\cooktime{40 minutes}
\dishtype{}
\begin{ingreds}
    6 tbsp butter
    1 tbsp olive oil
    1 medium onion, diced
    1 carrot, diced
    1 stick of celery, diced
    1 red bell pepper, diced
    3 cloves of garlic, diced
    1 can of San Marzano crushed tomatoes (28 oz)
    chicken stock (to desired thickness)
    1 cup heavy cream
    $\frac{1}{2}$ cup sherry (optional)
    1 tsp black pepper
    salt to taste
    sugar to taste
    basil (chiffonade)
    $\frac{1}{2}$ cup balsamic vinegar   
\end{ingreds}
\begin{method}
    Add the butter and oil to a saut\'e pan.\par
    Saut\'e (with a pinch of salt) the onion, carrot, celery, and bell pepper on medium-high heat until the onions are translucent.\par
    Add the garlic and cook for an additional 30 seconds.\par
    Deglaze the pan with the sherry (or chicken stock).\par
    Add the tomatoes and chicken stock and bring to a simmer.\par
    Add the black pepper and sugar until the desired taste is reached.\par
    Simmer for $>$30 min.\par
    Transfer to a blender and blend on high until completely \linebreak smooth.\par
    Transfer back to the saut\'e pan and add in the heavy cream.\par
    Add the balsamic vinegar to a non-stick pan on medium heat to make a reduction.\par
    Serve with the chiffonade basil and the balsamic reduction.
\end{method}

% Chicken Enchilada Soup
\recipe[This just might cure your cold.]{Chicken Enchilada Soup}{Shug}
\serves{4}
\preptime{15 minutes}
\cooktime{1 minutes}
\dishtype{}
\begin{ingreds}
    2 tbsp olive oil
    1 small onion (diced)
    2 stalks celery (diced)
    1 garlic clove (sliced)
    1 can green chilis (4 oz)
    1 can Rotel tomatoes
    1 can beef broth (16 oz)
    1 can chicken broth (16 oz)
    1 can cream of chicken soup
    1$\frac{1}{2}$ cups water
    1 tbsp A1 steak sauce
    $\frac{1}{4}$ tsp pepper
    2 tbsp Worcestershire sauce
    1 tbsp cumin
    1 tbsp chili powder
    2--3 cups cooked white rice 
    1 rotisserie chicken (pulled) 
\end{ingreds}
\begin{method}[For cooked rice, see the recipe in the side dish section.]
    In a large pot, saut\'e the onion and garlic in olive oil.\par
    Add all of the ingredients to the pot, excluding the rice.\par
    Bring to a boil, then lower the heat and simmer for 1 hour.\par
    Add the cooked rice, and serve.
\end{method}

% Chicken Orzo Soup
\recipe[Why fuss with rice, if you can just use orzo?]{Chicken Orzo Soup}{Gage}
\serves{4}
\preptime{20 minutes}
\cooktime{45 minutes}
\dishtype{}
\begin{ingreds}
    1 tbsp olive oil
    6 garlic cloves (minced)
    1 yellow onion (diced)
    2 carrots (thinly sliced)
    2 celery stalks (chopped)
    1 tbsp fresh ginger
    1 tsp cayenne
    1 tsp ground turmeric
    6 cups chicken broth
    1 lb chicken breast/thighs
    1 tsp chopped rosemary
    1 tsp chopped thyme
    $\frac{1}{2}$ tsp salt
    black pepper (to taste)
    1 cup orzo pasta
    $\frac{2}{3}$ cup frozen peas
\end{ingreds}
\begin{method}[Feel free to substitute with pre-cooked chicken.]
    Place a large dutch oven over medium high heat.\par
    Add the olive oil once the pot is ripping hot and add in the chicken making sure not to crowd the pot.\par
    Get a good sear on all sides of the chicken and save for later. There should be a nice fond in the pan at this point.\par
    Add in the onion, carrots and celery.\par
    Cook for a few minutes until the onion becomes translucent.\par
    Add in the ginger, garlic, and turmeric and saut\'e for 30 seconds.\par
    Add in chicken broth, chicken breast from earlier, and seasonings.\par
    Bring soup to a boil then reduce heat to low and simmer until chicken is fully cooked.\par
    Once chicken is cooked, remove and transfer to a cutting board to shred.\par
    Add chicken back to the pot and stir in frozen peas and orzo.\par
    Cook for another 7--8 minutes until orzo is just barely tender.
\end{method}




%%%%%%%%%%%%%%%%%%%%%%%%%%%%%%%%%%%%%%%%%%%%%%%%%%%%%%%%%%%%%%

\recipeSection{Desserts}

%%%%%%%%%%%%%%%%%%%%%%%%%%%%%%%%%%%%%%%%%%%%%%%%%%%%%%%%%%%%%%

% Creme Brulee
\recipe[There's something wrong with you if you don't like this dish.]{Cr\'eme Br\^ul\'ee}{Gage}
\serves{4}
\preptime{20 minutes}
\cooktime{30--60 minutes}
\dishtype{}
\begin{ingreds}
    2 cups heavy cream
    5 egg yolks
    $\frac{1}{2}$ cup granulated sugar
    1 tsp vanilla extract
    $\frac{1}{4}$ tsp kosher salt
    sugar for topping
    Fruit of choice for topping
\end{ingreds}
\begin{method}[Preheat oven to 325 degrees. A torch will work better than a broiler for caramelizing the sugar topping.]
    In a saucepan, combine cream, vanilla bean, and salt, and cook over low heat just until beginning to steam. Let sit for 5--10 minutes.\par
	In a bowl, beat yolks and sugar together until light.\par
    Stir about a quarter of the cream into this mixture, then pour sugar-egg mixture into cream and stir.\par
    Pour into four 6-ounce ramekins and place ramekins in a baking dish\par
    Fill dish with water halfway up the sides of the dishes.\par
    Bake for 30 min to 1 hour, or until centers are barely set.\par
    Cool completely. Refrigerate for several hours or up to two days.\par
	When ready to serve, top each custard with about a teaspoon of sugar in a thin layer.\par
    Place ramekins in a broiler 2--3'' from heat source.\par
    Turn on the broiler and cook until the sugar melts and browns or even blackens a bit, about 5 minutes.\par
    Serve within two hours.
\end{method}






\end{document}