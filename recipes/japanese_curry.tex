\recipe[Try it with the pork katsu.]{Japanese Curry}{Gage}
\serves{4}
\preptime{20 minutes}
\cooktime{1 hour}
\dishtype{\vegetarian}
\begin{ingreds}
    $\frac{1}{4}$ cup of butter (1st)
	$\frac{1}{2}$ cup butter (2nd)
	1$\frac{1}{2}$ tbsp butter (3rd)
	3 onions (julienned)
	1-quart chicken stock
	4 dried shiitake mushroom 
	1 apple (peeled \& grated)
	1 tbsp tomato paste
	$\frac{1}{2}$ cup all-purpose flour
	1 tbsp garam masala
	$\frac{1}{4}$ cup curry powder
	2 tsp MSG
	1 tbsp soy sauce
	2 tbsp Worcestershire sauce
	1 tbsp honey or sugar        
\end{ingreds}
\begin{method}
    Add the 1st butter measurement to a large saucepot and melt over medium-high heat.\par
    Add the julienned onions and cook for 30 to 45 minutes, stirring often, and adjusting the temperature between medium and low accordingly; if onions start to stick to the bottom, add a splash of water to deglaze.\par
	Bring 1 cup of the chicken stock to a boil and add the shitake mushrooms; after two minutes, remove your mushrooms or until entirely hydrated. Set aside.\par
    Once the onions are a medium-dark color, add the grated apple and tomato paste to the onions. Stir and cook for 3--4 minutes or until the apple softens. Remove and place it aside.\par
	Over medium-high heat (using the same pot),  add the 2nd butter measurement.\par
    Once melted, add the flour, constantly whisk for 30 seconds.\par
    Add the garam masala (optional), curry powder, and MSG.\par
    Whisk vigorously, and let toast for about 1 minute.\par
	Add back the onions along with the soy sauce, Worcestershire sauce, honey (or sugar), and one quart of chicken shiitake stock, stirring often until thickened\par
    Let simmer for 2 minutes.\par
	Blend the mix until completely smooth, then add the 3rd butter measurement while blending.\par
    Serve with rice and pork katsu.
\end{method}