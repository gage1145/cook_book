\recipe[A more classic use of spaghetti.]{Spaghetti \& Meatballs}{Gage}
\serves{4}
\preptime{30 minutes}
\cooktime{30 minutes}
\dishtype{\main}
\dishother{}
\begin{ingreds}
    \ingredients[Meatballs]
        $\frac{1}{4}$ lb chopped mortadella
        1 lb ground beef
        1 tsp (2g) finely ground fennel
        1$\frac{3}{4}$ tsp (12g) fine sea salt
        3 cloves garlic (diced)
        $\frac{1}{4}$ cup (20g) grated parmesan
        black pepper to taste
        $\frac{1}{2}$ cup (35g) bread crumbs
        1 whole egg
    \columnbreak{}
    \ingredients[Sauce]
        $\frac{1}{4}$ cup extra virgin olive oil
        4 cloves garlic, thinly sliced
        1 tsp (2g) red pepper flakes
        1 can peeled tomatoes (28 oz)
        1 bunch of basil
        grated parmesan for serving
        chiffonade basil for serving
        1 package spaghetti
    
\end{ingreds}
\begin{method}[Prepare the meatballs before making the sauce.]
    In a bowl, place your chopped mortadella, combine it with the ground beef, fennel seed, salt, garlic, parmesan cheese, and black pepper, mix it well, and add the panko and eggs and mix again until emulsified and tacky.\par
    Use a large cookie scoop to make the meatballs, around twenty, place them on a sheet tray and roll them into balls.\par
    In a large saut\'ee pan, over medium-high, pour enough olive oil to cover the bottom of the pan; once the oil is hot, add all the meatballs in one single layer, and sear for about two minutes, flip and sear one or two more sides, until golden brown (it's okay if there are not cooked all the way).\par
    Remove the meatballs from the pan and reserve.\par

    In the same pan where the meatballs were cooked, reduce the heat to medium, add the garlic, and saut\'e for about 5 minutes.\par
    Add the pepper flakes, saut\'e for thirty seconds and add your crushed tomatoes.\par
    Stir in a pinch of sugar (or to desired sweetness).\par
    Add the meatballs back into the pan, bring to a simmer, and reduce the heat to medium-low and simmer for five to eight minutes.\par
    Halfway through this process, add the basil leaves and let simmer until the meatballs cook all the way through.\par

    Place spaghetti in a pot of boiling water that has been seasoned generously with salt.\par
    Cook according to package instructions or until done.\par
    Using tongs, pick up the pasta, let it drain slightly, add it to your sauce until all your pasta has been added.\par
    Place one portion of pasta in a shallow bowl, pour some sauce on top and two or three meatballs, grate some fresh parmesan, and finally, some chiffonade fresh basil.
\end{method}