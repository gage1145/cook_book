\recipe[Make this for every party you throw.]{Mojo Pulled Pork}{Gage}
\serves{4}
\preptime{20 minutes}
\cooktime{3$\frac{1}{2}$--4 hours}
\dishtype{\main}
\begin{ingreds}
    1 pork shoulder
    Chicken stock (if needed)
    Kosher salt (to taste)
    1 onion (quartered)
    2 heads of garlic (peeled)
    2 oranges (juice and zest)
    3 limes (juice and zest)
    2 tbsp (7g) oregano leaves
    $\frac{1}{2}$ bunch of mint leaves
    1 tbsp (5g) ground cumin
    2 serrano chilies
    1 cup extra virgin olive oil
    1$\frac{1}{2}$ tbsp kosher salt
    1 cup fresh lime juice
    1 cup fresh orange juice or pineapple juice
\end{ingreds}
\begin{method}[Marinate overnight; Preheat the oven to \temp{400} on day of cooking.]
    In a blender, add the onion, garlic, orange zest, lime zest, oregano leaves, mint leaves, cumin, serranos, olive oil, salt, lime juice, and orange juice.\par
    Blend together on high speed until completely smooth.\par
    Reserve 1 cup of the marinade to use for dipping.\par
    With a paring knife, poke deep holes all over the pork shoulder.\par
    In a large resealable bag, place the pork shoulder, and pour in the remaining mojo marinade to cover the meat. Seal the bag and marinate in the refrigerator overnight.\par
    Remove the pork from the marinade and place in a dutch oven.\par
    Pour in all of the marinade. The marinade should come about halfway up the pot, but if not, add a little bit of chicken stock.\par
    Braise the pork, uncovered, for 20 minutes. Then reduce the temperature to 350°F and cook until the internal temperature of the pork is 200°F ($~$3$\frac{1}{2}$--4 hours) flipping occasionally.\par
    Remove the pork from the dutch oven and place on a cutting board to cool.\par
    In a large bowl, using two forks, shred the meat.\par
    Toss together using the braising liquid to coat the meat to your desired level of fattiness.\par
    Season to taste with salt.    
\end{method}